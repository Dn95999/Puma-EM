\chapter{Method of moments solution terms}
\label{annex:MoM}

%
\par
In this section we are going to detail how to compute the terms of the MoM matrices that appear in (\ref{eqn:EFIE-O_PEC_2}) and (\ref{eqn:MFIE-O_PEC_2}) for PEC targets, namely
\begin{gather} 
D_{mn} = \int_{D_m}\vect{g}_m \arg{\vect{r}} \cdot \left( \left(\nabla \nabla \cdot + k^2\right) \int_{D_n} G\left(\vect{r}, \vect{r}'\right) \current{f}_n\left(\vect{r}'\right) d\vect{r}' \right) d\vect{r} \label{eqn:MoM terms I_1}\\
K_{mn} = \int_{D_m} \vect{g}_m \arg{\vect{r}} \cdot \left(\int_{D_n} \nabla G\left(\vect{r}, \vect{r}'\right) \times \current{f}_n\left(\vect{r}'\right) d\vect{r}'\right) d\vect{r} \label{eqn:MoM terms I_2}\\
J_{mn} = \int_{D_m} \frac{1}{2} \vect{g}_m \arg{\vect{r}} \cdot \vect{\hat{n}} \times \current{f}_{n}\arg{\vect{r}}d\vect{r}. \label{eqn:MoM terms I_3}
\end{gather}
As usual, the homogeneous space Green's function $G = \frac{e^{-jk\left|\vect{r}-\vect{r}' \right|}}{4 \pi \left|\vect{r}-\vect{r}' \right|}$, and $k = \omega \sqrt{\varepsilon \mu} = \omega \sqrt{\varepsilon_0 \varepsilon_r \mu_0 \mu_r}$.

\section{Computation of $D_{mn}$}
%
\par
First let us take care of the derivative term. It is written as:
\begin{equation}
\nabla \nabla \cdot \int_{D_n} G\arg{\vect{r},\vect{r}'} \vect{f}_n\arg{\vect{r}'} \; \ud S',
\end{equation}
where the integration is performed on the domain $D_n$ of the RWG basis function $\vect{f}_n$. By using the fact that $\nabla \cdot G = -\nabla' \cdot G$ and basic vector identities, we can write the following sequence of equalities:
\begin{equation*}
\begin{split}
\nabla \nabla \cdot \int_{D_n} G \vect{f}_n \; \ud S' &= \nabla \int_{D_n} \nabla \cdot \left(G \vect{f}_n \right) \; \ud S' \\
&= \nabla \int_{D_n} \nabla G \cdot \vect{f}_n \; \ud S' \\
&= -\nabla \int_{D_n} \nabla' G \cdot \vect{f}_n \; \ud S' \\
&= -\nabla \int_{D_n} \left[\nabla' \cdot \left(G \vect{f}_n \right) - G \nabla_S^\prime \cdot \vect{f}_n\right]\; \ud S'\\
&= -\nabla \oint_{\partial D_n} \uvect{m}_n \cdot G \vect{f}_n \; \ud l + \nabla \int_{D_n} G \nabla_S^\prime \cdot \vect{f}_n  \; \ud S'
\end{split}
\end{equation*}
where $\partial D_n$ denotes integration on the border of $D_n$. Due to the properties of the RWG basis functions given at section \ref{subsubsec:Basis functions: definition and properties}, the contour integral yields zero. We can therefore write that 
\begin{equation}
\nabla \nabla \cdot \int_{D_n} G \vect{f}_n \; \ud S' = \nabla \int_{D_n} G \nabla_S^\prime \cdot \vect{f}_n  \; \ud S',
\end{equation}
which allows us to decompose $D_{mn}$ as follows:
\begin{equation}
D_{mn} = \underbrace{\int_{D_m} \vect{g}_m\arg{\vect{r}} \cdot \left(\nabla \int_{D_n} G\arg{\vect{r},\vect{r}'} \; \nabla_S^\prime \cdot \vect{f}_n  \; d \vect{r}'\right) \ud \vect{r} }_{\triangleq D_{mn,1}} + k^2\underbrace{\int_{D_m}\vect{g}_m \arg{\vect{r}} \cdot \left( \int_{D_n} G\arg{\vect{r},\vect{r}'} \vect{f}_n\arg{\vect{r}}\right) d\vect{r}}_{\triangleq D_{mn,2}}.
\end{equation}


\subsection{Testing with $\vect{g}_m = \vect{f}_m$: $D_{mn}^t$}
\subsubsection{$D_{mn,1}^t$}
%
\par
First we modify integral $D_{mn,1}^t$. The gradient applied to the inside inner product can be applied on the testing function by integrating by parts and by applying Gauss' divergence theorem \cite{Rao_82, Taskinen_03}:
\begin{equation} \label{eqn:Gauss' divergence theorem}
\begin{split}
\int_{D_m} \vect{f}_m \cdot \nabla \phi \; \ud S &= \int_{D_m} \nabla \cdot \left( \vect{f}_m \phi \right) \; \ud S - \int_{D_m} \phi \nabla_S \cdot \vect{f}_m \; \ud S \\
&= \int_{\partial D_m} \left(\vect{f}_m \phi \right) \cdot \uvect{m} \; \ud l - \int_{D_m} \phi \nabla_S \cdot \vect{f}_m \; \ud S
\end{split}
\end{equation}
where $\phi$ is a scalar function, $\partial D_m$ is the contour of $D_m$ and $\uvect{m} = \uvect{l} \times \uvect{n}$ is the normal to the contour of the basis function domain. The contour integral yields zero, due to the fact that, on the edges of the RWG basis function domain $D_m$, $\vect{f}_m$ is parallel to $\uvect{l}_m$. We therefore have that
\begin{equation}
D_{mn,1}^t = -\int_{D_m}  \nabla_S \cdot \vect{f}_m\arg{\vect{r}}  \left( \int_{D_n} G\arg{\vect{r},\vect{r}'} \; \nabla_S^\prime \cdot \vect{f}_n  \; d \vect{r}'\right) \ud \vect{r}.
\end{equation}
%
\par
From the divergence properties of the RWG basis functions given by (\ref{eqn:RWG divergence}), it is easy to see that $D_{mn,1}^t$ is constituted by the following sum:
\begin{equation} \label{eqn:MoM scalar potential term}
D_{mn,1}^t = \sum_{p, \, T_m^p \in D_m} \; \sum_{q, \, T_n^q \in D_n} -4 C_m^p C_n^q \int_{T_m^p} \int_{T_n^q} G \; \ud S' \ud S
\end{equation}
with $C_m^p = \frac{S_m^p l_m}{2 A_m^p}$, where $S_m^p$ is the sign of test function $m$ in triangle $p$, $l_m$ is the length of edge $m$, and $A_m^p$ is the area of triangle $T_m^p$. The sums are performed on $p$ and $q$ for which $T_m^p$ and $T_n^q$ are the triangles that form half of $D_m$ and $D_n$ respectively. We can note that the term under the integration sign is independent upon the basis function. This constatation leads us to remark that, instead of evaluating the MoM integrals RWG-wise, we will perform them triangle-wise, and contribute back the terms into the MoM matrix with the appropriate coefficients for all basis and test functions that pertain to the two triangles \cite{Rao_82}. Since for closed surfaces, there are three basis functions per triangle, and because the majority of the time is spent on these integrals, performing them triangle-wise will save a non-negligible amount of computation time. In the remainder of the text, the two summation symbols will be dropped for clarity.
%
\par
One more comment must be made about (\ref{eqn:MoM scalar potential term}). $G$ is singular when $\vect{r} = \vect{r}'$, and this renders the numerical integration very imprecise if $T_m^p$ is close to $T_n^q$. This singularity must be properly extracted. This is done thoroughly in \cite{Taskinen_03} and will not be discussed here.

\subsubsection{$D_{mn,2}^t$}
%
\par
Now let us develop $D_{mn,2}^t$. From definition (\ref{eqn:RWG definition}) of the RWG basis functions, it is immediate to see that $D_{mn,2}^t$ is constituted by combinations of the following term:
\begin{multline}  \label{eqn:MoM vector potential term}
C_m^p C_n^q \int_{T_m^p} \left(\vect{r}-\vect{r}_m^p \right) \cdot \int_{T_n^q} G \left(\vect{r}'-\vect{r}_n^q \right) \ud S' \ud S = C_m^p C_n^q \Bigg[ \int_{T_m^p} \vect{r} \cdot \int_{T_n^q} G \; \vect{r}' \ud S' \ud S  - \vect{r}_n^q \cdot \int_{T_m^p} \vect{r} \int_{T_n^q} G \; \ud S' \ud S \\
- \vect{r}_m^p \cdot \int_{T_m^p} \int_{T_n^q} G \; \vect{r}' \ud S' \ud S + \vect{r}_m^p \cdot \vect{r}_n^q \int_{T_m^p} \int_{T_n^q} G \; \ud S' \ud S \Bigg]
\end{multline}
where $\vect{r}_m^p$ is the vector position of the node belonging to triangle $p$ and opposite to edge $m$. The four integrals on the right-hand side of the above equation are independent of the edges, and each can again be performed triangle-wise. Only their recombination is edge-dependent.

\subsection{Testing with $\vect{g}_m = \uvect{n} \times \vect{f}_m$: $D_{mn}^n$}

\subsubsection{$D_{mn,1}^n$}
%
\par
Rewriting explicitly $D_{mn,1}^n$ with $\vect{g}_m = \uvect{n} \times \vect{f}_m$, we can see that it will be a combination of the following term:
\begin{equation}
2 C_m^p C_n^q \int_{T_m^p} \big(\uvect{n}_m^p \times \left(\vect{r}-\vect{r}_m^p \right) \big) \cdot \int_{T_n^q} \nabla G \; \ud S' \ud S
\end{equation}
This time we cannot move the gradient onto the testing function in $D_{mn,1}^n$, because $\uvect{n} \times \vect{f}_m$ is not continuous on the triangle pair that forms $D_m$ \cite{Taskinen_03}, as the triangles normals differ. The integral may be further decomposed as:
\begin{equation} \label{eqn:n_x_fm_grad_G_terms}
\begin{split}
\int_{T_m^p} \big(\uvect{n}_m^p \times \left(\vect{r}-\vect{r}_m^p \right) \big) \cdot \int_{T_n^q} \nabla G \; \ud S' \ud S & = \int_{T_m^p} \left(\uvect{n}_m^p \times \vect{r} \right) \cdot \int_{T_n^q} \nabla G \; \ud S' \ud S - \left(\uvect{n}_m^p \times \vect{r}_m^p\right) \cdot \int_{T_m^p} \int_{T_n^q} \nabla G \; \ud S' \ud S \\
& = \uvect{n}_m^p \cdot \left(\int_{T_m^p} \vect{r} \times \int_{T_n^q} \nabla G \; \ud S' \ud S - \vect{r}_m^p \times \int_{T_m^p} \int_{T_n^q} \nabla G \; \ud S' \ud S\right).
\end{split}
\end{equation}
%
\par
The kernel involved in the terms contained in (\ref{eqn:n_x_fm_grad_G_terms}) is highly singular, as it involves an integral of a $1/R^2$ singularity, in contrast with an integral of a $1/R$ singularity usually involved with $G$ \cite{Taskinen_03}. When domains of basis functions $m$ and $n$ are ``sufficiently far away'' from each other, a regular numerical integration of the terms appearing in (\ref{eqn:n_x_fm_grad_G_terms}) should not cause any trouble. But if the test and basis function overlap, we will not be able to extract analytically the singularity, and a numerical method will have to evaluate a great number of times the integrand for obtaining a precise value of the integral. 
%
\par
However, an elegant transformation may be applied to the integrand by noting that
\begin{equation}
\begin{split}
\int_{T_m^p} \big(\uvect{n}_m^p \times \left(\vect{r}-\vect{r}_m^p \right) \big) \cdot \int_{T_n^q} \nabla G \; \ud S' \ud S &= \uvect{n}_m^p \cdot \int_{T_m^p} \left(\vect{r}-\vect{r}_m^p \right) \times \int_{T_n^q} \nabla G \; \ud S' \ud S \\
&= - \uvect{n}_m^p \cdot \left[ \int_{T_m^p} \int_{T_n^q} \nabla \times \big( G \left( \vect{r}-\vect{r}_m^p \right)\big) \; \ud S' \ud S - \int_{T_m^p} \int_{T_n^q} G \nabla \times \left( \vect{r}-\vect{r}_m^p \right) \; \ud S' \ud S\right] \\
&=  - \uvect{n}_m^p \cdot \int_{\partial T_m^p} \big(\uvect{m}_m^p \times  \left( \vect{r}-\vect{r}_m^p \right) \big) \int_{T_n^q}  G \; \ud S' \ud l 
\end{split}
\end{equation}
where use of identity $\nabla \times \left(a \vect{b}\right) = a \nabla \times \vect{b} - \vect{b} \times \nabla a$ has been made. The last equality is due to the fact that the rotational of the position vector $ \left( \vect{r}-\vect{r}_m^p \right)$ is zero. By noting that $\uvect{m} = \uvect{l} \times \uvect{n}$, we finally obtain
\begin{equation}
\begin{split}
\int_{T_m^p} \big(\uvect{n}_m^p  \times \left(\vect{r}-\vect{r}_m^p \right) \big) \cdot \int_{T_n^q} \nabla G \; \ud S' \ud S &= \int_{\partial T_m^p} -\uvect{l}_m^p \cdot  \left( \vect{r}-\vect{r}_m^p \right)  \int_{T_n^q}  G \; \ud S' \ud l \\
& = \int_{\partial T_m^p} -\uvect{l}_m^p \cdot \vect{r}  \int_{T_n^q}  G \; \ud S' \ud l + \vect{r}_m^p \cdot \int_{\partial T_m^p}\uvect{l}_m^p  \int_{T_n^q}  G \; \ud S' \ud l
\end{split}
\end{equation}
which is a more elegant form than its counterpart (23) of \cite{Taskinen_03}. Singularity $\frac{1}{R^2}$ has been reduced to $\frac{1}{R}$ and can therefore be analytically extracted even for overlapping basis and test functions. 

\subsubsection{$D_{mn,2}^n$}
%
\par
Let us now develop $D_{mn,2}^n$. We will have the following terms:
\begin{multline}
C_m^p C_n^q \int_{T_m^p} \big( \uvect{n}_m^p \times \left(\vect{r} - \vect{r}_m^p \right)\big) \cdot \int_{T_n^q} G \left(\vect{r}'-\vect{r}_n^q \right) \; \ud S' \ud S \\
= C_m^p C_n^q  \Bigg[\int_{T_m^p} \left(\uvect{n}_m^p \times \vect{r} \right) \cdot \int_{T_n^q} G \; \vect{r}' \; \ud S' \ud S - \vect{r}_n^q \cdot \left(\uvect{n} \times \int_{T_m^p} \vect{r} \int_{T_n^q} G \; \ud S' \ud S \right)  \\
- \left(\uvect{n}_m^p \times \vect{r}_m^p\right)  \cdot \int_{T_m^p} \int_{T_n^q} G \; \vect{r}'\; \ud S' \ud S + \vect{r}_n^q \cdot \left(\uvect{n}_m^p \times \vect{r}_m^p\right) \int_{T_m^p} \int_{T_n^q} G \; \ud S' \ud S \Bigg].
\end{multline}
These terms do not pose any particular problems, and some of them are already present in (\ref{eqn:MoM vector potential term}).

\section{Computation of $K_{mn}$}
%
\par
Let us rewrite $K_{mn} = \int_{D_m} \vect{g}_m \cdot \int_{D_n} \nabla G \times \vect{f}_n \; \ud S' \ud S$. Note that, if the two triangles $T_m^p$ and $T_n^q$ that form respectively half of $D_m$ and half of $D_n$ are coplanar, the corresponding contribution $K_{mn}^{pq} = 0$, since in that case $\nabla G$ is contained in the same plane as $\vect{f}_n$ and $\vect{g}_m$, and therefore $\nabla G \times \vect{f}_n$ and $\vect{g}_m$ are perpendicular. \textit{A fortiori}, $K_{mn}^{pq}$ will be zero if $T_m^p = T_n^q$.

\subsection{Testing with $\vect{g}_m = \vect{f}_m$: $K_{mn}^t$}
%
\par
After replacing $\left(\vect{r}' - \vect{r}_n^q\right)$ by $\left(\vect{r}' - \vect{r}\right) + \left(\vect{r} - \vect{r}_m^p\right) + \left(\vect{r}_m^p - \vect{r}_n^q\right)$ and performing a few manipulations, it can be shown that (\ref{eqn:MoM terms I_2}) implies terms of type (see (20) of \cite{Taskinen_03}):
\begin{multline} \label{eqn:G_HJ_fs_testing_decomposition}
C_m^p C_n^q \int_{T_m^p} \left(\vect{r} - \vect{r}_m^p \right) \cdot \left[-\left(\vect{r}_m^p - \vect{r}_n^q\right) \times \int_{T_n^q} \nabla G \; \ud S'\right] \; \ud S \\
= C_m^p C_n^q \left(\vect{r}_m^p - \vect{r}_n^q\right) \cdot \left[\int_{T_m^p} \vect{r} \times \int_{T_n^q} \nabla G \; \ud S' \ud S - \vect{r}_m^p \times \int_{T_m^p} \int_{T_n^q}\nabla G \; \ud S' \ud S \right].
\end{multline}

\subsection{Testing with $\vect{g}_m = \uvect{n} \times \vect{f}_m$: $K_{mn}^n$}
%
\par
In this case, by replacing $\left(\vect{r}'-\vect{r}_n^q \right)$ by $\left(\vect{r}'-\vect{r} \right) + \left(\vect{r}-\vect{r}_n^q \right)$, (\ref{eqn:MoM terms I_2}) implies terms of type:
\begin{multline} \label{eqn:G_HJ_fs_n_x_RWG_testing_decomposition}
C_m^p C_n^q \int_{T_m^p} \big( \uvect{n}_m^p \times \left(\vect{r} - \vect{r}_m^p \right) \big) \cdot \left[-\left(\vect{r} - \vect{r}_n^q\right) \times \int_{T_n^q} \nabla G \; \ud S'\right] \; \ud S \\ 
= -C_m^p C_n^q \left\{ \int_{T_m^p} \left(\uvect{n}_m^p \times \vect{r}\right) \cdot \left[\vect{r} \times \int_{T_n^q} \nabla G \; \ud S' \right] \ud S \right. \\
+ \vect{r}_n^q \cdot \int_{T_m^p} \left(\uvect{n}_m^p \times \vect{r}\right) \times \int_{T_n^q} \nabla G \; \ud S' \ud S  - \left(\uvect{n}_m^p\times\vect{r}_m^p \right) \cdot \int_{T_m^p} \vect{r} \times \int_{T_n^q} \nabla G \; \ud S' \ud S \\
\left. - \vect{r}_n^q \cdot \left[\left(\uvect{n} \times \vect{r}_m^p \right) \times \int_{T_m^p} \int_{T_n^q}\nabla G \; \ud S' \ud S\right]\right\}.
\end{multline}
Note that two terms of (\ref{eqn:G_HJ_fs_n_x_RWG_testing_decomposition}) are present in (\ref{eqn:G_HJ_fs_testing_decomposition}).

\section{Computation of $J_{mn}$}

\subsection{Testing with $\vect{g}_m = \vect{f}_m$: $J_{mn}^t$}
%
\par
$J_{mn}$ is nonzero only for overlapping triangles $p$ and $q$, and will involve combinations of terms of the type
\begin{equation} \label{eqn:MMPIE_J_RWG_testing_diag_term}
C_m^p C_n^p \int_{T_m^p} \left(\vect{r}-\vect{r}_m^p\right) \cdot \left(\uvect{n}_m^p \times \left(\vect{r}-\vect{r}_n^p\right) \right) \; \ud S =  - C_m^p C_n^p \uvect{n}_m^p \cdot \left[\left(\vect{r}_n^p-\vect{r}_m^p\right) \times \int_{T_m^p} \vect{r} \; \ud S + \left(\vect{r}_m^p \times \vect{r}_n^p\right) \int_{T_m^p} 1 \; \ud S \right].
\end{equation}

\subsection{Testing with $\vect{g}_m = \uvect{n} \times \vect{f}_m$: $J_{mn}^n$}
%
\par
We can immediately write that
\begin{equation}
\innerprod{\uvect{n} \times \vect{f}_m}{;}{\uvect{n} \times \vect{f}_n}{D_m} = \innerprod{\vect{f}_m}{;}{\vect{f}_n}{D_m} 
\end{equation}
where the right-hand term can be further decomposed using basic vector formulas as
\begin{equation} \label{eqn:MMPIE_J_n_x_RWG_testing_diag_term}
C_m^p C_n^p \int_{T_m^p} \left(\vect{r}-\vect{r}_m^p \right) \cdot \left(\vect{r}-\vect{r}_n^p\right) \; \ud S = C_m^p C_n^p \left[ \int_{T_m^p} \left|\vect{r}\right|^2 \; \ud S - \left(\vect{r}_m^p + \vect{r}_n^p \right) \cdot \int_{T_m^p} \vect{r} \; \ud S + \left(\vect{r}_m^p \cdot \vect{r}_n^p \right) \int_{T_m^p} 1 \; \ud S\right].
\end{equation}


\section{Computation of the MoM excitation vectors}
\label{app:Computation of the MoM excitation vectors}
%
\par
The nonzero terms involved in the computation of the excitation vectors $\Array{V}{}{E}$ and $\Array{V}{}{H}$ have the following generic form:
\begin{equation}
V_m^{P} = -\int_{D_m}\vect{g}_m \arg{\vect{r}} \cdot \field{P}^\text{inc} d\vect{r}
\end{equation}
where $\vect{P}$ can be $\vect{E}$ or $\vect{H}$. If tested with $\vect{g}_m = \vect{f}_m$, we immediately have that
\begin{equation}
-\int_{D_m}\vect{f}_m \arg{\vect{r}} \cdot \field{P}^\text{inc} d\vect{r} = -\frac{l_m}{2 A_m^+}\int_{T_m^+} \left(\vect{r} - \vect{r}_m^+ \right) \cdot \vect{P}^\text{inc} \; \ud S + \frac{l_m}{2 A_m^-}\int_{T_m^-} \left(\vect{r} - \vect{r}_m^- \right) \cdot \vect{P}^\text{inc} \; \ud S.
\end{equation}
If tested with $\vect{g}_m = \uvect{n} \times \vect{f}_m$, we have 
\begin{equation}
-\int_{D_m} \left(\uvect{n}\times\vect{f}_m\arg{\vect{r}}\right)  \cdot \field{P}^\text{inc} d\vect{r} = -\frac{l_m}{2 A_m^+}\int_{T_m^+} \big( \uvect{n}_m^+ \times \left(\vect{r} - \vect{r}_m^+ \right) \big) \cdot \vect{P}^\text{inc} \; \ud S + \frac{l_m}{2 A_m^-}\int_{T_m^-} \big( \uvect{n}_m^- \times \left(\vect{r} - \vect{r}_m^- \right)\big) \cdot \vect{P}^\text{inc} \; \ud S.
\end{equation}

\section{Summary}
\begin{table}[h!]
\centering
\begin{tabular}{|c||c|c|c|}
  \hline
& $D_{mn,1}$ & $D_{mn,2}$ & $K_{mn}$ \\
  \hline
\multirow{4}{*}{$\vect{g}_m = \vect{f}_m$} & \multirow{4}{*}{$-4 C_m^p C_n^q \int_{T_m} \int_{T_n} G$} & $C_m^p C_n^q  \Bigg[\int_{T_m} \vect{r} \cdot \int_{T_n} G \; \vect{r}'$ & \\
& & $- \vect{r}_n^q \cdot \int_{T_m} \vect{r} \int_{T_n} G$ & $C_m^p C_n^q \left(\vect{r}_m^p - \vect{r}_n^q\right) \cdot \Bigg[\int_{T_m} \vect{r} \times \int_{T_n} \nabla G$ \\
& & $- \vect{r}_m^p  \cdot \int_{T_m} \int_{T_n} G \; \vect{r}'$ & $- \vect{r}_m^p \times \int_{T_m} \int_{T_n}\nabla G \Bigg]$\\
& & $+ \vect{r}_m^p \cdot \vect{r}_n^q \int_{T_m} \int_{T_n} G \Bigg]$ & \\
  \hline
\multirow{4}{*}{$\vect{g}_m = \uvect{n}\times\vect{f}_m$} &  & $C_m^p C_n^q  \Bigg[\int_{T_m} \left(\uvect{n} \times \vect{r} \right) \cdot \int_{T_n} G \; \vect{r}'$ & $-C_m^p C_n^q \Bigg\{ \int_{T_m} \left(\uvect{n} \times \vect{r}\right) \cdot \left[\vect{r} \times \int_{T_n} \nabla G \right]$\\
& $2C_m^p C_n^q\uvect{n} \cdot \Big(\int_{T_m} \vect{r} \times \int_{T_n} \nabla G$ & $- \vect{r}_n^q \cdot \left(\uvect{n} \times \int_{T_m} \vect{r} \int_{T_n} G \right) $ & $+ \vect{r}_n^q \cdot \int_{T_m} \left(\uvect{n} \times \vect{r}\right) \times \int_{T_n} \nabla G$\\
& $- \vect{r}_m^p \times \int_{T_m^p} \int_{T_n^q} \nabla G \Big)$ & $- \left(\uvect{n} \times \vect{r}_m^p\right)  \cdot \int_{T_m} \int_{T_n} G \; \vect{r}'$ & $- \left(\uvect{n}\times\vect{r}_m^p \right) \cdot \int_{T_m} \vect{r} \times \int_{T_n} \nabla G$\\
& & $+ \vect{r}_n^q \cdot \left(\uvect{n} \times \vect{r}_m^p\right) \int_{T_m} \int_{T_n} G \Bigg]$ & $-\vect{r}_n^q \cdot \left[\left(\uvect{n} \times \vect{r}_m^p \right) \times \int_{T_m} \int_{T_n}\nabla G \right]\Bigg\}$\\
  \hline

\end{tabular}
\end{table}



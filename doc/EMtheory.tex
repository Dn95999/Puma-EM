\documentclass[a4paper,10pt]{book}
\usepackage{t1enc}
\usepackage[english]{babel}
\usepackage{calc}
\usepackage{amsmath,amssymb}
\usepackage{epsfig}
\usepackage{fourier}
%\usepackage{utopia}
%\usepackage{times}
\usepackage[T1]{fontenc}
\usepackage{a4wide}
\usepackage{url}
\usepackage{fancyhdr}
\usepackage{array,longtable}
\usepackage{color}
\usepackage{listings}

\setcounter{secnumdepth}{3}
\setcounter{tocdepth}{3}

\definecolor{orange}{rgb}{1,0.5,0}
\definecolor{blue}{rgb}{0.3,0.3,0.9}

\newcommand{\file}[1] {\textcolor{blue}{\textsf{#1}}}
\newcommand{\parameter}[1] {\textcolor{orange}{\textsf{#1}}}

\newcommand{\field}[1]{\mathbf{#1}}
\newcommand{\current}[1]{\mathbf{#1}}
\newcommand{\vect}[1]{\mathbf{#1}}
\newcommand{\operator}[1]{\mathbf{#1}}

\title{Electromagnetic Theory}
\author{Idesbald van den Bosch}

\begin{document}
\maketitle
\tableofcontents
\bibliographystyle{unsrt}

\chapter{The integral equations and their combinations}
%
\par
We are going to derive the integral equations for a dielectric body immersed in a homogeneous medium. We have the following parameters:
\begin{description}
\item [$c = 2.99792458e8$] speed of light in vacuum
\item [$\mu_0 = 4.0e-7 \pi$] permeability of vacuum
\item [$\mu_{r, i}$] relative permeability of medium $i$
\item [$\mu_i = \mu_0\mu_{r, i}$] permeability of medium $i$
\item [$\varepsilon_0 = 1.0/(c^2 \mu_0)$] permittivity of vacuum
\item [$\varepsilon_{r, i}$] relative permittivity of medium $i$
\item [$\varepsilon_i = \varepsilon_0\varepsilon_{r, i}$] permittivity of medium $i$
\item [$k_i = \omega \sqrt{\mu_i \varepsilon_i}$] wavenumber of medium $i$ 
\item [$\omega = 2 \pi f$] where $f$ is the frequency.
\end{description}

\section{Integral equations outside of inhomogeneity}
%
\par
According to \cite[Eqs. 1-41, 1-45, 1-47 and 1-48]{Harrington2001} and to \cite[Eqs. 1.9.2 and 1.9.3]{Orfanidis2002}, the time-harmonic Maxwell's equations can be written as (an $e^{j \omega t}$ time dependence is assumed and suppressed):
\begin{align}
\field{H} &= \current{J} +  j\omega \epsilon \field{E} \label{eqn:Maxwell-H} \\
\field{E} &= -\current{M} -  j\omega \mu \field{H}. \label{eqn:Maxwell-E}
\end{align}
Based on these equations, and according to \cite{Ides2006}, the Electric field outside of an inhomogeneity bounded by a surface $S$ as in figure \ref{fig:inhomogeneity} can be written as:
\begin{equation}\label{eqn:electricFieldInt}
\field{E}_1\left(\vect{r}\right) = \field{E}^\text{inc}\left(\vect{r}\right) + \frac{\nabla \nabla \cdot + k_1^2}{j \omega \epsilon_1} \int_S G_1(\vect{r}, \vect{r}') \left(\vect{\hat{n}} \times \field{H}_1\left(\vect{r}'\right) \right) d\vect{r}' - \nabla \times \int_S G_1(\vect{r}, \vect{r}') \left( \field{E}_1\left(\vect{r}'\right) \times \vect{\hat{n}} \right) d\vect{r}'
\end{equation}
where $\vect{\hat{n}}$ is the outward normal to the surface $S$, and with the free space Green's function:
\begin{equation*}
G_1(\vect{r}, \vect{r}') = \frac{e^{-jk_1 \left|\vect{r} - \vect{r}'  \right|}}{4 \pi \left|\vect{r} - \vect{r}'  \right|}.
\end{equation*}
\begin{figure}
\setlength{\unitlength}{1cm}
\centering
\begin{picture}(3,3)
%\put(0,0){\dashbox{0.2}(3,3)}
\put(1,1){\circle{2.5}}
\put(1,1){\text{2}}
\put(1.7,1.7){\text{1}}
\put(1,2){\vector(0,1){1}}
\put(1.1,2.1){\text{$\vect{\hat{n}}$}}
\end{picture}
\caption{Inhomogeneity with random surface}
\label{fig:inhomogeneity}
\end{figure}
%
\par
Applying duality (by using (\ref{eqn:Maxwell-H}) and (\ref{eqn:Maxwell-E}): $\field{E} \rightarrow \field{H}$, $\field{H} \rightarrow -\field{E}$, $\epsilon \rightarrow \mu$, $\current{J} \rightarrow \current{M}$, $\current{M} \rightarrow -\current{J}$) to (\ref{eqn:electricFieldInt}), we obtain the Magnetic-field integral equation:
\begin{equation}\label{eqn:magneticFieldInt}
\field{H}_1\left(\vect{r}\right) = \field{H}^\text{inc}\left(\vect{r}\right) + \frac{\nabla \nabla \cdot + k_1^2}{j \omega \mu_1} \int_S G_1(\vect{r}, \vect{r}') \left(\field{E}_1\left(\vect{r}'\right) \times \vect{\hat{n}}\right) d\vect{r}' + \nabla \times \int_S G_1(\vect{r}, \vect{r}') \left(  \vect{\hat{n}} \times \field{H}_1\left(\vect{r}'\right) \right) d\vect{r}'.
\end{equation}
%
\par
Defining the equivalent currents:
\begin{align}
\current{J}_{S,1} &\triangleq \vect{\hat{n}} \times \field{H}_1 \\
\current{M}_{S,1} &\triangleq \field{E}_1 \times \vect{\hat{n}} 
\end{align}
we can rewrite (\ref{eqn:magneticFieldInt}) as follows:
\begin{equation}\label{eqn:magneticFieldInt-2}
\field{H}_1\left(\vect{r}\right) = \field{H}^\text{inc}\left(\vect{r}\right) + \frac{\nabla \nabla \cdot + k_1^2}{j \omega \mu_1} \int_S G_1(\vect{r}, \vect{r}') \current{M}_{S,1}\left(\vect{r}'\right) d\vect{r}' + \int_S \nabla G_1(\vect{r}, \vect{r}') \times \current{J}_{S,1}\left(\vect{r}'\right) d\vect{r}' 
\end{equation}
where the vector calculus identity $\nabla \times \left( G(\vect{r}, \vect{r}') \current{J}\left(\vect{r}'\right)\right) = \nabla G(\vect{r}, \vect{r}') \times \current{J}\left(\vect{r}'\right)$ has been used.
%
\par
Evaluating (\ref{eqn:magneticFieldInt-2}) at a point $\vect{r}$ on the surface $S$, and taking the Cauchy principal value \cite{Arnoldus2011} yields:
\begin{equation}\label{eqn:magneticFieldInt-3}
\field{H}_1\left(\vect{r}\right) = \field{H}^\text{inc}\left(\vect{r}\right) + \frac{\nabla \nabla \cdot + k_1^2}{j \omega \mu_1} \int_S G_1(\vect{r}, \vect{r}') \current{M}_{S,1}\left(\vect{r}'\right) d\vect{r}' -\frac{1}{2} \vect{\hat{n}} \times \current{J}_{S,1}\left(\vect{r}\right) + \int_S \nabla G_1(\vect{r}, \vect{r}') \times \current{J}_{S,1}\left(\vect{r}'\right) d\vect{r}'.
\end{equation}
But, since $\current{J}_{S,1} \triangleq \vect{\hat{n}} \times \field{H}_1$, we have that $\field{H}_1 = -\vect{\hat{n}} \times \current{J}_{S,1}$, and (\ref{eqn:magneticFieldInt-3}) is rewritten as:
\begin{equation}\label{eqn:magneticFieldInt-4}
\boxed{
\field{H}^\text{inc}\left(\vect{r}\right) = - \frac{1}{2} \vect{\hat{n}} \times \current{J}_{S,1}\left(\vect{r}\right) - \frac{\nabla \nabla \cdot + k_1^2}{j \omega \mu_1} \int_S G_1(\vect{r}, \vect{r}') \current{M}_{S,1}\left(\vect{r}'\right) d\vect{r}'  - \int_S \nabla G_1(\vect{r}, \vect{r}') \times \current{J}_{S,1}\left(\vect{r}'\right) d\vect{r}'}
\end{equation}
which is known as the outside magnetic field integral equation. It can be conveniently rewritten in a more compact form as follows:
\begin{equation}\label{eqn:MFIE-O}
\boxed{
\field{H}^\text{inc} = - \frac{1}{2} \vect{\hat{n}} \times \current{J}_{S,1} - \frac{1}{j \omega \mu_1} D_1 \left( \current{M}_{S,1}\right) - K_1 \left(\current{J}_{S,1}\right)
} \qquad MFIE-O
\end{equation}
The outside electric field integral equation is then simply obtained by duality:
\begin{equation}\label{eqn:EFIE-O}
\boxed{
\field{E}^\text{inc} =  \frac{1}{2} \vect{\hat{n}} \times \current{M}_{S,1} - \frac{1}{j \omega \epsilon_1} D_1 \left( \current{J}_{S,1}\right) + K_1 \left(\current{M}_{S,1}\right)
} \qquad EFIE-O
\end{equation}


\section{Integral equations inside of inhomogeneity}
%
\par
According to \cite{Ides2006}, the Electric field inside of the inhomogeneity of figure \ref{fig:inhomogeneity} can be written as:
\begin{equation}\label{eqn:electricFieldIntInside}
\field{E}_2\left(\vect{r}\right) = \frac{\nabla \nabla \cdot + k_2^2}{j \omega \epsilon_2} \int_S G_2(\vect{r}, \vect{r}') \left(-\vect{\hat{n}} \times \field{H}_2\left(\vect{r}'\right) \right) d\vect{r}' - \nabla \times \int_S G_2(\vect{r}, \vect{r}') \left( \field{E}_2\left(\vect{r}'\right) \times \left(-\vect{\hat{n}} \right)\right) d\vect{r}'.
\end{equation}
And by duality the MFIE:
\begin{equation}\label{eqn:magneticFieldIntInside}
\field{H}_2\left(\vect{r}\right) = \frac{\nabla \nabla \cdot + k_2^2}{j \omega \mu_2} \int_S G_2(\vect{r}, \vect{r}') \left(\vect{\hat{n}} \times \field{E}_2\left(\vect{r}'\right) \right) d\vect{r}' - \nabla \times \int_S G_2(\vect{r}, \vect{r}') \left( \field{H}_2\left(\vect{r}'\right) \times \left(-\vect{\hat{n}} \right)\right) d\vect{r}'.
\end{equation}
By defining (remember that $\vect{\hat{n}}$ points outwards):
\begin{align}
\current{J}_{S,2} &\triangleq (-\vect{\hat{n}}) \times \field{H}_2 \\
\current{M}_{S,2} &\triangleq \field{E}_2 \times (-\vect{\hat{n}}) 
\end{align}
we rewrite (\ref{eqn:magneticFieldIntInside}) as 
\begin{equation}\label{eqn:magneticFieldIntInside-2}
\field{H}_2\left(\vect{r}\right) = \frac{\nabla \nabla \cdot + k_2^2}{j \omega \mu_2} \int_S G_2(\vect{r}, \vect{r}') \current{M}_{S,2}\left(\vect{r}'\right) d\vect{r}' + \int_S \nabla G_2(\vect{r}, \vect{r}') \times \current{J}_{S,2}\left(\vect{r}'\right) d\vect{r}'.
\end{equation}
Evaluating (\ref{eqn:magneticFieldIntInside-2}) at a point $\vect{r}$ on the surface $S$, and taking the Cauchy principal value \cite{Arnoldus2011} yields:
\begin{equation}\label{eqn:magneticFieldIntInside-3}
\field{H}_2\left(\vect{r}\right) = \frac{\nabla \nabla \cdot + k_2^2}{j \omega \mu_2} \int_S G_2(\vect{r}, \vect{r}') \current{M}_{S,2}\left(\vect{r}'\right) d\vect{r}' + \frac{1}{2} \vect{\hat{n}} \times \current{J}_{S,2}\left(\vect{r}\right) +  \int_S \nabla G_2(\vect{r}, \vect{r}') \times \current{J}_{S,2}\left(\vect{r}'\right) d\vect{r}'.
\end{equation}
Since $\current{J}_{S,2} \triangleq (-\vect{\hat{n}}) \times \field{H}_2$, we have $\vect{\hat{n}} \times \current{J}_{S,2} = \field{H}_2$, and (\ref{eqn:magneticFieldIntInside-3}) is recast as
\begin{equation}\label{eqn:magneticFieldIntInside-4}
\begin{split}
0 &= - \frac{1}{2} \vect{\hat{n}} \times \current{J}_{S,2}\left(\vect{r}\right) + \frac{\nabla \nabla \cdot + k_2^2}{j \omega \mu_2} \int_S G_2(\vect{r}, \vect{r}') \current{M}_{S,2}\left(\vect{r}'\right) d\vect{r}' +  \int_S \nabla G_2(\vect{r}, \vect{r}') \times \current{J}_{S,2}\left(\vect{r}'\right) d\vect{r}' \\
&=  - \frac{1}{2} \vect{\hat{n}} \times \current{J}_{S,2} + \frac{1}{j \omega \mu_2} D_2\left(\current{M}_{S,2}\right) +  K_2\left( \current{J}_{S,2}\right).
\end{split}
\end{equation}

%
\par
Since the boundary conditions impose that $\field{E}_1 = \field{E}_2$ and $\field{H}_1 = \field{H}_2$, it follows from the equivalent current definitions that $\current{M}_{S,2} = -\current{M}_{S,1}$ and $\current{J}_{S,2} = -\current{J}_{S,1}$. Hence the last equation can be rewritten as follows:
\begin{equation}\label{eqn:MFIE-I}
\boxed{
0 = - \frac{1}{2} \vect{\hat{n}} \times \current{J}_{S,1} + \frac{1}{j \omega \mu_2} D_2\left(\current{M}_{S,1}\right) +  K_2\left( \current{J}_{S,1}\right)
} \qquad MFIE-I
\end{equation}
and the EFIE-I is obtained by duality:
\begin{equation}\label{eqn:EFIE-I}
\boxed{
0 = \frac{1}{2} \vect{\hat{n}} \times \current{M}_{S,1} + \frac{1}{j \omega \epsilon_2} D_2\left(\current{J}_{S,1}\right) -  K_2\left( \current{M}_{S,1}\right)
} \qquad EFIE-I
\end{equation}

\section{Combining the integral equations}
%
\par
Let's first recall the integral equations:
\begin{equation*}
\boxed{
\field{H}^\text{inc} = - \frac{1}{2} \vect{\hat{n}} \times \current{J}_{S,1} - \frac{1}{j \omega \mu_1} D_1 \left( \current{M}_{S,1}\right) - K_1 \left(\current{J}_{S,1}\right)
} \qquad MFIE-O \tag{\ref{eqn:MFIE-O} recalled}
\end{equation*}
\begin{equation*}
\boxed{
\field{E}^\text{inc} =  \frac{1}{2} \vect{\hat{n}} \times \current{M}_{S,1} - \frac{1}{j \omega \epsilon_1} D_1 \left( \current{J}_{S,1}\right) + K_1 \left(\current{M}_{S,1}\right)
} \qquad EFIE-O \tag{\ref{eqn:EFIE-O} recalled}
\end{equation*}
\begin{equation*}
\boxed{
0 = - \frac{1}{2} \vect{\hat{n}} \times \current{J}_{S,1} + \frac{1}{j \omega \mu_2} D_2\left(\current{M}_{S,1}\right) +  K_2\left( \current{J}_{S,1}\right)
} \qquad MFIE-I \tag{\ref{eqn:MFIE-I} recalled}
\end{equation*}
\begin{equation*}
\boxed{
0 = \frac{1}{2} \vect{\hat{n}} \times \current{M}_{S,1} + \frac{1}{j \omega \epsilon_2} D_2\left(\current{J}_{S,1}\right) -  K_2\left( \current{M}_{S,1}\right)
} \qquad EFIE-I \tag{\ref{eqn:EFIE-I} recalled}
\end{equation*}
with
\begin{align*}
D_i\left(\current{X}\right) &= \left(\nabla \nabla \cdot + k_i^2\right) \int_{S} G_i\left(\vect{r}, \vect{r}'\right) \current{X}\left(\vect{r}'\right) d\vect{r}' \\
K_i\left(\current{X}\right) &= \int_S \nabla G_i\left(\vect{r}, \vect{r}'\right) \times \current{X}\left(\vect{r}'\right) d\vect{r}'.
\end{align*}


\subsection{PMCHWT}
%
\par
For this formulation we combine the outside and inside on a per-field basis: EFIE-O (\ref{eqn:EFIE-O}) with EFIE-I (\ref{eqn:EFIE-I}), and MFIE-O (\ref{eqn:MFIE-O}) with MFIE-I (\ref{eqn:MFIE-I}).
\begin{eqnarray}
EFIE: & \boxed{
\left[\field{E}^\text{inc} =  -\frac{1}{j \omega \epsilon_1} \operator{D}_{1}\left(\current{J}_{S,1}\right) - \frac{1}{j \omega \epsilon_2} \operator{D}_{2}\left(\current{J}_{S,1}\right) + \operator{K}_{1}\left(\current{M}_{S,1}\right) + \operator{K}_{2}\left(\current{M}_{S,1}\right) \right]_\text{tan} } \\
HFIE: & \boxed{
\left[\field{H}^\text{inc} = - \operator{K}_{1}\left(\current{J}_{S,1}\right) - \operator{K}_{2}\left(\current{J}_{S,1}\right) - \frac{1}{j \omega \mu_1}\operator{D}_{1}\left(\current{M}_{S,1}\right) - \frac{1}{j \omega \mu_2}\operator{D}_{2}\left(\current{M}_{S,1}\right)  \right]_\text{tan} }.
\end{eqnarray}
In matrix form, we have that
\begin{equation}
\left[
\begin{matrix}
  - \left(\operator{D}_{\varepsilon, 1} + \operator{D}_{\varepsilon, 2} \right) & \operator{K}_{1} + \operator{K}_{2} \\
  -\left(\operator{K}_{1} + \operator{K}_{2} \right) & - \left(\operator{D}_{\mu, 1} + \operator{D}_{\mu, 2}\right)
\end{matrix}
\right]
\cdot 
\left[
\begin{matrix}
  \current{J}_{S,1} \\
  \current{M}_{S,1}  
\end{matrix}
\right]
=
\left[
\begin{matrix}
  \field{E}^\text{inc} \\
  \field{H}^\text{inc}
\end{matrix}
\right].
\end{equation}

\subsection{CFIE}
%
\par
For this formulation we combine the fields on an outside and inside basis: EFIE-1 (\ref{eqn:EFIE_1}) with HFIE-1 (\ref{eqn:HFIE_1}), and EFIE-2 (\ref{eqn:EFIE_2}) with HFIE-2 (\ref{eqn:HFIE_2}).
\begin{eqnarray}
CFIE-1: & \boxed{
\left[\alpha \field{E}^\text{inc} + \left(1-\alpha \right) \frac{1}{\eta}\field{H}^\text{inc} =  - \alpha \left(\operator{D}_{\varepsilon, 1}\left(\current{J}_{S,1}\right) - \operator{K}_{1}\left(\current{J}_{S,1}\right) \right) + \operator{K}_{1}\left(\current{M}_{S,1}\right) + \operator{D}_{\mu, 1}\left(\current{M}_{S,1}\right) \right]_\text{tan} } \\
CFIE-2: & \boxed{
\left[\field{H}^\text{inc} = - \operator{K}_{1}\left(\current{J}_{S,1}\right) - \operator{K}_{2}\left(\current{J}_{S,1}\right) - \operator{D}_{\mu, 1}\left(\current{M}_{S,1}\right) - \operator{D}_{\mu, 2}\left(\current{M}_{S,1}\right)  \right]_\text{tan} }.
\end{eqnarray}
In matrix form, we have that
\begin{equation}
\left[
\begin{matrix}
  - \left(\operator{D}_{\varepsilon, 1} + \operator{D}_{\varepsilon, 2} \right) & \operator{K}_{1} + \operator{K}_{2} \\
  -\left(\operator{K}_{1} + \operator{K}_{2} \right) & - \left(\operator{D}_{\mu, 1} + \operator{D}_{\mu, 2}\right)
\end{matrix}
\right]
\cdot 
\left[
\begin{matrix}
  \current{J}_{S,1} \\
  \current{M}_{S,1}  
\end{matrix}
\right]
=
\left[
\begin{matrix}
  \field{E}^\text{inc} \\
  \field{H}^\text{inc}
\end{matrix}
\right].
\end{equation}

\bibliography{references}
\end{document}

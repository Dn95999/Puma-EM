\documentclass[a4paper,10pt]{book}
\usepackage{t1enc}
\usepackage[english]{babel}
\usepackage{calc}
\usepackage{amsmath,amssymb}
\usepackage{epsfig}
\usepackage{fourier}
%\usepackage{utopia}
%\usepackage{times}
\usepackage[T1]{fontenc}
\usepackage{a4wide}
\usepackage{url}
\usepackage{fancyhdr}
\usepackage{array,longtable}
\usepackage{color}
\usepackage{listings}

\setcounter{secnumdepth}{3}
\setcounter{tocdepth}{3}

\definecolor{orange}{rgb}{1,0.5,0}
\definecolor{blue}{rgb}{0.3,0.3,0.9}

\newcommand{\file}[1] {\textcolor{blue}{\textsf{#1}}}
\newcommand{\parameter}[1] {\textcolor{orange}{\textsf{#1}}}

\newcommand{\field}[1]{\mathbf{#1}}
\newcommand{\current}[1]{\mathbf{#1}}
\newcommand{\vect}[1]{\mathbf{#1}}
\newcommand{\uvect}[1]{\ensuremath{\mathbf{\Hat{#1}}}}
\newcommand{\operator}[1]{\mathbf{#1}}
\newcommand{\innerprod}[4]{\ensuremath{\left\langle #1 #2 #3 \right\rangle}_{#4}}
\newcommand{\Array}[3]{\ensuremath{\underline{#1} \thinspace \negthinspace _{#2} ^{#3}}}
\newcommand{\AArray}[3]{\ensuremath{\underline{\underline{#1}} \thinspace \negthinspace _{#2} ^{#3}}}
\newcommand{\ud}[0]{\text{d}}
\renewcommand{\arg}[1]{\ensuremath{\!\left(#1\right)}}
\newcommand{\dyadic}[3]{\ensuremath{\overline{\pmb{\mathcal{#1}}}\thinspace \negthinspace _{#2} ^{#3}}}


\title{Electromagnetic Theory}
\author{Idesbald van den Bosch}

\begin{document}
\maketitle
\tableofcontents
\bibliographystyle{unsrt}

\chapter{The integral equations and their combinations}
%
\par
We are going to derive the integral equations for a dielectric body immersed in a homogeneous medium. We have the following parameters:
\begin{description}
\item [$c = 2.99792458e8$] speed of light in vacuum
\item [$\mu_0 = 4.0e-7 \pi$] permeability of vacuum
\item [$\mu_{r, i}$] relative permeability of medium $i$
\item [$\mu_i = \mu_0\mu_{r, i}$] permeability of medium $i$
\item [$\varepsilon_0 = 1.0/(c^2 \mu_0)$] permittivity of vacuum
\item [$\varepsilon_{r, i}$] relative permittivity of medium $i$
\item [$\varepsilon_i = \varepsilon_0\varepsilon_{r, i}$] permittivity of medium $i$
\item [$k_i = \omega \sqrt{\mu_i \varepsilon_i}$] wavenumber of medium $i$ 
\item [$\omega = 2 \pi f$] where $f$ is the frequency.
\end{description}

\section{Integral equations outside of inhomogeneity}
%
\par
According to \cite[Eqs. 1-41, 1-45, 1-47 and 1-48]{Harrington2001} and to \cite[Eqs. 1.9.2 and 1.9.3]{Orfanidis2002}, the time-harmonic Maxwell's equations can be written as (an $e^{j \omega t}$ time dependence is assumed and suppressed):
\begin{align}
\field{H} &= \current{J} +  j\omega \epsilon \field{E} \label{eqn:Maxwell-H} \\
\field{E} &= -\current{M} -  j\omega \mu \field{H}. \label{eqn:Maxwell-E}
\end{align}
Based on these equations, and according to \cite{Ides2006}, the Electric field outside of an inhomogeneity bounded by a surface $S$ as in figure \ref{fig:inhomogeneity} can be written as:
\begin{equation}\label{eqn:electricFieldInt}
\field{E}_1\left(\vect{r}\right) = \field{E}^\text{inc}\left(\vect{r}\right) + \frac{\nabla \nabla \cdot + k_1^2}{j \omega \epsilon_1} \int_S G_1(\vect{r}, \vect{r}') \left(\vect{\hat{n}} \times \field{H}_1\left(\vect{r}'\right) \right) d\vect{r}' - \nabla \times \int_S G_1(\vect{r}, \vect{r}') \left( \field{E}_1\left(\vect{r}'\right) \times \vect{\hat{n}} \right) d\vect{r}'
\end{equation}
where $\vect{\hat{n}}$ is the outward normal to the surface $S$, and with the free space Green's function:
\begin{equation*}
G_1(\vect{r}, \vect{r}') = \frac{e^{-jk_1 \left|\vect{r} - \vect{r}'  \right|}}{4 \pi \left|\vect{r} - \vect{r}'  \right|}.
\end{equation*}
\begin{figure}
\setlength{\unitlength}{1cm}
\centering
\begin{picture}(3,3)
%\put(0,0){\dashbox{0.2}(3,3)}
\put(1,1){\circle{2.5}}
\put(1,1){\text{2}}
\put(1.7,1.7){\text{1}}
\put(1,2){\vector(0,1){1}}
\put(1.1,2.1){\text{$\vect{\hat{n}}$}}
\end{picture}
\caption{Inhomogeneity with random surface}
\label{fig:inhomogeneity}
\end{figure}
%
\par
Applying duality (by using (\ref{eqn:Maxwell-H}) and (\ref{eqn:Maxwell-E}): $\field{E} \rightarrow \field{H}$, $\field{H} \rightarrow -\field{E}$, $\epsilon \rightarrow \mu$, $\current{J} \rightarrow \current{M}$, $\current{M} \rightarrow -\current{J}$) to (\ref{eqn:electricFieldInt}), we obtain the Magnetic-field integral equation:
\begin{equation}\label{eqn:magneticFieldInt}
\field{H}_1\left(\vect{r}\right) = \field{H}^\text{inc}\left(\vect{r}\right) + \frac{\nabla \nabla \cdot + k_1^2}{j \omega \mu_1} \int_S G_1(\vect{r}, \vect{r}') \left(\field{E}_1\left(\vect{r}'\right) \times \vect{\hat{n}}\right) d\vect{r}' + \nabla \times \int_S G_1(\vect{r}, \vect{r}') \left(  \vect{\hat{n}} \times \field{H}_1\left(\vect{r}'\right) \right) d\vect{r}'.
\end{equation}
%
\par
Defining the equivalent currents:
\begin{align}
\current{J}_{S,1} &\triangleq \vect{\hat{n}} \times \field{H}_1 \\
\current{M}_{S,1} &\triangleq \field{E}_1 \times \vect{\hat{n}} 
\end{align}
we can rewrite (\ref{eqn:magneticFieldInt}) as follows:
\begin{equation}\label{eqn:magneticFieldInt-2}
\field{H}_1\left(\vect{r}\right) = \field{H}^\text{inc}\left(\vect{r}\right) + \frac{\nabla \nabla \cdot + k_1^2}{j \omega \mu_1} \int_S G_1(\vect{r}, \vect{r}') \current{M}_{S,1}\left(\vect{r}'\right) d\vect{r}' + \int_S \nabla G_1(\vect{r}, \vect{r}') \times \current{J}_{S,1}\left(\vect{r}'\right) d\vect{r}' 
\end{equation}
where the vector calculus identity $\nabla \times \left( G(\vect{r}, \vect{r}') \current{J}\left(\vect{r}'\right)\right) = \nabla G(\vect{r}, \vect{r}') \times \current{J}\left(\vect{r}'\right)$ has been used.
%
\par
Evaluating (\ref{eqn:magneticFieldInt-2}) at a point $\vect{r}$ on the surface $S$, and taking the Cauchy principal value \cite{Arnoldus2011}, \cite[p.~139]{Morita_90} yields:
\begin{equation}\label{eqn:magneticFieldInt-3}
\field{H}_1\left(\vect{r}\right) = \field{H}^\text{inc}\left(\vect{r}\right) + \frac{\nabla \nabla \cdot + k_1^2}{j \omega \mu_1} \int_S G_1(\vect{r}, \vect{r}') \current{M}_{S,1}\left(\vect{r}'\right) d\vect{r}' -\frac{1}{2} \vect{\hat{n}} \times \current{J}_{S,1}\left(\vect{r}\right) + \int_S \nabla G_1(\vect{r}, \vect{r}') \times \current{J}_{S,1}\left(\vect{r}'\right) d\vect{r}'.
\end{equation}
But, since $\current{J}_{S,1} \triangleq \vect{\hat{n}} \times \field{H}_1$, we have that $\field{H}_1 = -\vect{\hat{n}} \times \current{J}_{S,1}$, and (\ref{eqn:magneticFieldInt-3}) is rewritten as:
\begin{equation}\label{eqn:magneticFieldInt-4}
\boxed{
\field{H}^\text{inc}\left(\vect{r}\right) = - \frac{1}{2} \vect{\hat{n}} \times \current{J}_{S,1}\left(\vect{r}\right) - \frac{\nabla \nabla \cdot + k_1^2}{j \omega \mu_1} \int_S G_1(\vect{r}, \vect{r}') \current{M}_{S,1}\left(\vect{r}'\right) d\vect{r}'  - \int_S \nabla G_1(\vect{r}, \vect{r}') \times \current{J}_{S,1}\left(\vect{r}'\right) d\vect{r}'}
\end{equation}
which is known as the outside magnetic field integral equation. It can be conveniently rewritten in a more compact form as follows:
\begin{equation}\label{eqn:MFIE-O}
\boxed{
\field{H}^\text{inc} = - \frac{1}{2} \vect{\hat{n}} \times \current{J}_{S,1} - \frac{1}{j \omega \mu_1} \operator{D}_1 \left( \current{M}_{S,1}\right) - \operator{K}_1 \left(\current{J}_{S,1}\right)
} \qquad MFIE-O
\end{equation}
The outside electric field integral equation is then simply obtained by duality:
\begin{equation}\label{eqn:EFIE-O}
\boxed{
\field{E}^\text{inc} =  \frac{1}{2} \vect{\hat{n}} \times \current{M}_{S,1} - \frac{1}{j \omega \epsilon_1} \operator{D}_1 \left( \current{J}_{S,1}\right) + \operator{K}_1 \left(\current{M}_{S,1}\right)
} \qquad EFIE-O
\end{equation}


\section{Integral equations inside of inhomogeneity}
%
\par
According to \cite{Ides2006}, the Electric field inside of the inhomogeneity of figure \ref{fig:inhomogeneity} can be written as:
\begin{equation}\label{eqn:electricFieldIntInside}
\field{E}_2\left(\vect{r}\right) = \frac{\nabla \nabla \cdot + k_2^2}{j \omega \epsilon_2} \int_S G_2(\vect{r}, \vect{r}') \left(-\vect{\hat{n}} \times \field{H}_2\left(\vect{r}'\right) \right) d\vect{r}' - \nabla \times \int_S G_2(\vect{r}, \vect{r}') \left( \field{E}_2\left(\vect{r}'\right) \times \left(-\vect{\hat{n}} \right)\right) d\vect{r}'.
\end{equation}
And by duality the MFIE:
\begin{equation}\label{eqn:magneticFieldIntInside}
\field{H}_2\left(\vect{r}\right) = \frac{\nabla \nabla \cdot + k_2^2}{j \omega \mu_2} \int_S G_2(\vect{r}, \vect{r}') \left(\vect{\hat{n}} \times \field{E}_2\left(\vect{r}'\right) \right) d\vect{r}' - \nabla \times \int_S G_2(\vect{r}, \vect{r}') \left( \field{H}_2\left(\vect{r}'\right) \times \left(-\vect{\hat{n}} \right)\right) d\vect{r}'.
\end{equation}
By defining (remember that $\vect{\hat{n}}$ points outwards):
\begin{align}
\current{J}_{S,2} &\triangleq (-\vect{\hat{n}}) \times \field{H}_2 \\
\current{M}_{S,2} &\triangleq \field{E}_2 \times (-\vect{\hat{n}}) 
\end{align}
we rewrite (\ref{eqn:magneticFieldIntInside}) as 
\begin{equation}\label{eqn:magneticFieldIntInside-2}
\field{H}_2\left(\vect{r}\right) = \frac{\nabla \nabla \cdot + k_2^2}{j \omega \mu_2} \int_S G_2(\vect{r}, \vect{r}') \current{M}_{S,2}\left(\vect{r}'\right) d\vect{r}' + \int_S \nabla G_2(\vect{r}, \vect{r}') \times \current{J}_{S,2}\left(\vect{r}'\right) d\vect{r}'.
\end{equation}
Evaluating (\ref{eqn:magneticFieldIntInside-2}) at a point $\vect{r}$ on the surface $S$, and taking the Cauchy principal value \cite{Arnoldus2011} yields:
\begin{equation}\label{eqn:magneticFieldIntInside-3}
\field{H}_2\left(\vect{r}\right) = \frac{\nabla \nabla \cdot + k_2^2}{j \omega \mu_2} \int_S G_2(\vect{r}, \vect{r}') \current{M}_{S,2}\left(\vect{r}'\right) d\vect{r}' + \frac{1}{2} \vect{\hat{n}} \times \current{J}_{S,2}\left(\vect{r}\right) +  \int_S \nabla G_2(\vect{r}, \vect{r}') \times \current{J}_{S,2}\left(\vect{r}'\right) d\vect{r}'.
\end{equation}
Since $\current{J}_{S,2} \triangleq (-\vect{\hat{n}}) \times \field{H}_2$, we have $\vect{\hat{n}} \times \current{J}_{S,2} = \field{H}_2$, and (\ref{eqn:magneticFieldIntInside-3}) is recast as
\begin{equation}\label{eqn:magneticFieldIntInside-4}
\begin{split}
0 &= - \frac{1}{2} \vect{\hat{n}} \times \current{J}_{S,2}\left(\vect{r}\right) + \frac{\nabla \nabla \cdot + k_2^2}{j \omega \mu_2} \int_S G_2(\vect{r}, \vect{r}') \current{M}_{S,2}\left(\vect{r}'\right) d\vect{r}' +  \int_S \nabla G_2(\vect{r}, \vect{r}') \times \current{J}_{S,2}\left(\vect{r}'\right) d\vect{r}' \\
&=  - \frac{1}{2} \vect{\hat{n}} \times \current{J}_{S,2} + \frac{1}{j \omega \mu_2} \operator{D}_2\left(\current{M}_{S,2}\right) +  \operator{K}_2\left( \current{J}_{S,2}\right).
\end{split}
\end{equation}

%
\par
Since the boundary conditions impose that $\field{E}_1 = \field{E}_2$ and $\field{H}_1 = \field{H}_2$, it follows from the equivalent current definitions that $\current{M}_{S,2} = -\current{M}_{S,1}$ and $\current{J}_{S,2} = -\current{J}_{S,1}$. Hence the last equation can be rewritten as follows:
\begin{equation}\label{eqn:MFIE-I}
\boxed{
0 = - \frac{1}{2} \vect{\hat{n}} \times \current{J}_{S,1} + \frac{1}{j \omega \mu_2} \operator{D}_2\left(\current{M}_{S,1}\right) +  \operator{K}_2\left( \current{J}_{S,1}\right)
} \qquad MFIE-I
\end{equation}
and the EFIE-I is obtained by duality:
\begin{equation}\label{eqn:EFIE-I}
\boxed{
0 = \frac{1}{2} \vect{\hat{n}} \times \current{M}_{S,1} + \frac{1}{j \omega \epsilon_2} \operator{D}_2\left(\current{J}_{S,1}\right) -  \operator{K}_2\left( \current{M}_{S,1}\right)
} \qquad EFIE-I
\end{equation}

\section{Combining the integral equations}
%
\par
Let's first recall the integral equations:
\begin{equation*}
\boxed{
\field{H}^\text{inc} = - \frac{1}{2} \vect{\hat{n}} \times \current{J}_{S,1} - \frac{1}{j \omega \mu_1} \operator{D}_1 \left( \current{M}_{S,1}\right) - \operator{K}_1 \left(\current{J}_{S,1}\right)
} \qquad MFIE-O \tag{\ref{eqn:MFIE-O} recalled}
\end{equation*}
\begin{equation*}
\boxed{
\field{E}^\text{inc} =  \frac{1}{2} \vect{\hat{n}} \times \current{M}_{S,1} - \frac{1}{j \omega \epsilon_1} \operator{D}_1 \left( \current{J}_{S,1}\right) + \operator{K}_1 \left(\current{M}_{S,1}\right)
} \qquad EFIE-O \tag{\ref{eqn:EFIE-O} recalled}
\end{equation*}
\begin{equation*}
\boxed{
0 = - \frac{1}{2} \vect{\hat{n}} \times \current{J}_{S,1} + \frac{1}{j \omega \mu_2} \operator{D}_2\left(\current{M}_{S,1}\right) +  \operator{K}_2\left( \current{J}_{S,1}\right)
} \qquad MFIE-I \tag{\ref{eqn:MFIE-I} recalled}
\end{equation*}
\begin{equation*}
\boxed{
0 = \frac{1}{2} \vect{\hat{n}} \times \current{M}_{S,1} + \frac{1}{j \omega \epsilon_2} \operator{D}_2\left(\current{J}_{S,1}\right) -  \operator{K}_2\left( \current{M}_{S,1}\right)
} \qquad EFIE-I \tag{\ref{eqn:EFIE-I} recalled}
\end{equation*}
with
\begin{align*}
\operator{D}_i\left(\current{X}\right) &= \left(\nabla \nabla \cdot + k_i^2\right) \int_{S} G_i\left(\vect{r}, \vect{r}'\right) \current{X}\left(\vect{r}'\right) d\vect{r}' \\
\operator{K}_i\left(\current{X}\right) &= \int_S \nabla G_i\left(\vect{r}, \vect{r}'\right) \times \current{X}\left(\vect{r}'\right) d\vect{r}'.
\end{align*}


\subsection{PMCHWT}
%
\par
For this formulation we combine the outside and inside on a per-field basis: EFIE-O (\ref{eqn:EFIE-O}) with EFIE-I (\ref{eqn:EFIE-I}), and MFIE-O (\ref{eqn:MFIE-O}) with MFIE-I (\ref{eqn:MFIE-I}).
\begin{eqnarray}
EFIE: & \boxed{
\left[\field{E}^\text{inc} =  -\frac{1}{j \omega \epsilon_1} \operator{D}_{1}\left(\current{J}_{S,1}\right) - \frac{1}{j \omega \epsilon_2} \operator{D}_{2}\left(\current{J}_{S,1}\right) + \operator{K}_{1}\left(\current{M}_{S,1}\right) + \operator{K}_{2}\left(\current{M}_{S,1}\right) \right]_\text{tan} } \\
HFIE: & \boxed{
\left[\field{H}^\text{inc} = - \operator{K}_{1}\left(\current{J}_{S,1}\right) - \operator{K}_{2}\left(\current{J}_{S,1}\right) - \frac{1}{j \omega \mu_1}\operator{D}_{1}\left(\current{M}_{S,1}\right) - \frac{1}{j \omega \mu_2}\operator{D}_{2}\left(\current{M}_{S,1}\right)  \right]_\text{tan} }.
\end{eqnarray}
In matrix form, we have that
\begin{equation}
\left[
\begin{matrix}
  -\frac{1}{j \omega \epsilon_1} \operator{D}_{1} - \frac{1}{j \omega \epsilon_2} \operator{D}_{2} & \operator{K}_{1} + \operator{K}_{2} \\
  -\left(\operator{K}_{1} + \operator{K}_{2} \right) & - \frac{1}{j \omega \mu_1}\operator{D}_{1} - \frac{1}{j \omega \mu_2}\operator{D}_{2}
\end{matrix}
\right]
\cdot 
\left[
\begin{matrix}
  \current{J}_{S,1} \\
  \current{M}_{S,1}  
\end{matrix}
\right]
=
\left[
\begin{matrix}
  \field{E}^\text{inc} \\
  \field{H}^\text{inc}
\end{matrix}
\right].
\end{equation}

\subsection{CFIE}
%
\par
For this formulation we combine the outside fields together, and the inside fields together: EFIE-O (\ref{eqn:EFIE-O}) with MFIE-O (\ref{eqn:MFIE-O}), and EFIE-I (\ref{eqn:EFIE-I}) with MFIE-I (\ref{eqn:MFIE-I}).
\begin{eqnarray}
CFIE-1: & \boxed{
\left[\alpha \field{E}^\text{inc} + \left(1-\alpha \right) \frac{1}{\eta}\field{H}^\text{inc} =  - \alpha \left(\operator{D}_{\varepsilon, 1}\left(\current{J}_{S,1}\right) - \operator{K}_{1}\left(\current{J}_{S,1}\right) \right) + \operator{K}_{1}\left(\current{M}_{S,1}\right) + \operator{D}_{\mu, 1}\left(\current{M}_{S,1}\right) \right]_\text{tan} } \\
CFIE-2: & \boxed{
\left[\field{H}^\text{inc} = - \operator{K}_{1}\left(\current{J}_{S,1}\right) - \operator{K}_{2}\left(\current{J}_{S,1}\right) - \operator{D}_{\mu, 1}\left(\current{M}_{S,1}\right) - \operator{D}_{\mu, 2}\left(\current{M}_{S,1}\right)  \right]_\text{tan} }.
\end{eqnarray}
In matrix form, we have that
\begin{equation}
\left[
\begin{matrix}
  - \left(\operator{D}_{\varepsilon, 1} + \operator{D}_{\varepsilon, 2} \right) & \operator{K}_{1} + \operator{K}_{2} \\
  -\left(\operator{K}_{1} + \operator{K}_{2} \right) & - \left(\operator{D}_{\mu, 1} + \operator{D}_{\mu, 2}\right)
\end{matrix}
\right]
\cdot 
\left[
\begin{matrix}
  \current{J}_{S,1} \\
  \current{M}_{S,1}  
\end{matrix}
\right]
=
\left[
\begin{matrix}
  \field{E}^\text{inc} \\
  \field{H}^\text{inc}
\end{matrix}
\right].
\end{equation}


\chapter{Solution of the integral equations for PEC}
%
\par
Let us concentrate on perfect electric conductor (PEC), a very important case in practice. In this case, the fields are zero inside of the inhomogeneity, magnetic surface currents are zero, and the integral equations are given by rewriting (\ref{eqn:EFIE-O}) and (\ref{eqn:MFIE-O}) as follows
\begin{equation}\label{eqn:EFIE-O_PEC}
\boxed{
\field{E}^\text{inc} =  - \frac{1}{j \omega \epsilon_1} \operator{D}_1 \left( \current{J}_{S}\right) 
} \qquad EFIE-O
\end{equation}
\begin{equation}\label{eqn:MFIE-O_PEC}
\boxed{
\field{H}^\text{inc} = - \frac{1}{2} \vect{\hat{n}} \times \current{J}_{S} - \operator{K}_1 \left(\current{J}_{S}\right)
} \qquad MFIE-O
\end{equation}
with
\begin{align*}
\operator{D}_1\arg{\current{J}_S} &= \left(\nabla \nabla \cdot + k_1^2\right) \int_{S} G_1\left(\vect{r}, \vect{r}'\right) \current{J}_S\left(\vect{r}'\right) d\vect{r}' \\
\operator{K}_1\left(\current{J}_S\right) &= \int_S \nabla G_1\left(\vect{r}, \vect{r}'\right) \times \current{J}_S\left(\vect{r}'\right) d\vect{r}'.
\end{align*}


\section{Method of moments formulation}

\subsection{Basis functions: definition and properties}
\label{subsubsec:Basis functions: definition and properties}
%
\par
First the surface of the body is discretized by triangular patches. Then the electric current are approximated by a sum of RWG \emph{basis functions} such that, for any point on the surface of the body, we have
\begin{equation}\label{eqn:current_decomposition}
\vect{J}\arg{\vect{r}} \simeq \sum_{n = 1}^{N} I_n \vect{f}_n \arg{\vect{r}}
\end{equation}
where $\vect{f}_n \arg{\vect{r}}$ are triangular edge-defined basis functions called RWG functions, first introduced by Rao, Wilton and Glisson in \cite{Rao_82}. $N$ is the number of basis functions. The $n^\text{th}$ RWG basis function is defined on the adjacent triangles associated with edge $n$, and is given by \cite{Michalski_90}:
\begin{equation} \label{eqn:RWG definition}
\vect{f}_n \arg{\vect{r}} \triangleq
\begin{cases}
\vect{f}_n^+ = \frac{l_n}{2 A_n^+} \left( \vect{r}^+ - \vect{r}_n^+\right), & \quad \text{$\vect{r}$ in $T_n^+$} \\
\vect{f}_n^- = -\frac{l_n}{2 A_n^-} \left( \vect{r}^- - \vect{r}_n^-\right), & \quad \text{$\vect{r}$ in $T_n^-$} \\
\vect{0}, & \quad \text{otherwise}
\end{cases}
\end{equation}
in which $l_n$ is the length of the edge, $A_n^\pm$ is the area of triangle $T_n^\pm$ and $\left( \vect{r}^\pm - \vect{r}_n^\pm\right)$ is the position vector in the triangle plane and relative to the node opposed to the edge.
%
\par
These RWG basis functions display properties that are very useful in the MoM, among which the most important are given hereafter \cite{Rao_82}:
\begin{enumerate}
\item the component of current normal to the $n^\text{th}$ edge is constant and continuous across the edge;
\item all edges of $T_n^+$ and $T_n^-$ are free of line charges;
\item the surface divergence of $\vect{f}_n$, proportional to the surface charge density associated with the basis element, is
\begin{equation} \label{eqn:RWG divergence}
\nabla_S \cdot \vect{f}_n \arg{\vect{r}} =
\begin{cases}
\frac{l_n}{A_n^+}, & \quad \text{$\vect{r}$ in $T_n^+$} \\
-\frac{l_n}{A_n^-}, & \quad \text{$\vect{r}$ in $T_n^-$} \\
0, & \quad \text{otherwise}.
\end{cases}
\end{equation}
The charge density is constant in each triangle and the total charge associated with the triangle pair $T_n^+$ and $T_n^-$ is zero;
\item a linear superposition of the three basis functions associated with a triangle can represent a linear current flowing through this triangle in an arbitrary direction.
\end{enumerate}

\subsection{Discretization and testing of the integral equations}
%
\par
With (\ref{eqn:current_decomposition}), (\ref{eqn:EFIE-O_PEC}) and (\ref{eqn:MFIE-O_PEC}) become:
\begin{equation}\label{eqn:EFIE-O_PEC_1}
\field{E}^\text{inc} =  - \frac{1}{j \omega \epsilon_1} \sum_{n=1}^{N} I_n \left(\nabla \nabla \cdot + k_1^2\right) \int_{D_n} G_1\left(\vect{r}, \vect{r}'\right) \current{f}_n\left(\vect{r}'\right) d\vect{r}'
\end{equation}
\begin{equation}\label{eqn:MFIE-O_PEC_1}
\field{H}^\text{inc} = - \frac{1}{2} \sum_{n=1}^{N} I_n \vect{\hat{n}} \times \current{f}_{n}\arg{\vect{r}} - \sum_{n=1}^{N} J_n \int_{D_n} \nabla G_1\left(\vect{r}, \vect{r}'\right) \times \current{f}_n\left(\vect{r}'\right) d\vect{r}'
\end{equation}
where $D_n$ is the domain of basis function $\vect{f}_n$.
%
\par
Let us now test the integral equations as per \cite{Har_68}, \cite{Rao_82}. The choice of the testing functions is widely debated in the literature, and is of paramount importance for obtaining a precise and well-defined numerical approximation of the solution. Testing is most often done with help of the RWG basis functions, in which case we are in a Galerkin numerical scheme.  However, testing with $\uvect{n} \times \text{RWG}$ functions must also be considered when using the CFIE. The testing function, denoted by $\vect{g}_m$, can therefore be:
\begin{equation}
\vect{g}_m \triangleq \vect{f}_m \qquad \text{or} \qquad \vect{g}_m \triangleq \uvect{n} \times \vect{f}_m \qquad \text{on} \; D_m
\end{equation}
where $\uvect{n}$ is the outward normal to the surface, and where $D_m$ is the domain of the test function $\vect{g}_m$, defined in the same manner as for $\vect{f}_m$.
%
\par
Testing (\ref{eqn:EFIE-O_PEC_1}) and (\ref{eqn:MFIE-O_PEC_1}) with $\vect{g}_m$ yields
\begin{equation}\label{eqn:EFIE-O_PEC_2}
\boxed{-\int_{D_m}\vect{g}_m \arg{\vect{r}} \cdot \field{E}^\text{inc} d\vect{r} =  \frac{1}{j \omega \epsilon_1} \sum_{n=1}^{N} I_n \left[\int_{D_m}\vect{g}_m \arg{\vect{r}} \cdot \left( \left(\nabla \nabla \cdot + k_1^2\right) \int_{D_n} G_1\left(\vect{r}, \vect{r}'\right) \current{f}_n\left(\vect{r}'\right) d\vect{r}' \right) d\vect{r} \right]}
\end{equation}
\begin{equation}\label{eqn:MFIE-O_PEC_2}
\boxed{-\int_{D_m}\vect{g}_m \arg{\vect{r}} \cdot \field{H}^\text{inc} d\vect{r} = \sum_{n=1}^{N} I_n \left[\int_{D_m} \frac{1}{2} \vect{g}_m \arg{\vect{r}} \cdot \vect{\hat{n}} \times \current{f}_{n}\arg{\vect{r}}d\vect{r} + \int_{D_m} \vect{g}_m \arg{\vect{r}} \cdot \left(\int_{D_n} \nabla G_1\left(\vect{r}, \vect{r}'\right) \times \current{f}_n\left(\vect{r}'\right) d\vect{r}'\right) d\vect{r} \right]}.
\end{equation}
We can recast the above equations in matrix form:
\begin{equation}
\underline{\underline{Z}}^{EJ} \underline{I} = \underline{V}^{E}, \quad \underline{\underline{Z}}^{HJ} \underline{I} = \underline{V}^{H}
\end{equation}
where
\begin{itemize}
\item $Z_{mn}^{EJ} = \frac{1}{j \omega \epsilon_1} \int_{D_m}\vect{g}_m \arg{\vect{r}} \cdot \left( \left(\nabla \nabla \cdot + k_1^2\right) \int_{D_n} G_1\left(\vect{r}, \vect{r}'\right) \current{f}_n\left(\vect{r}'\right) d\vect{r}' \right) d\vect{r}$
\item $V_m^{E} = -\int_{D_m}\vect{g}_m \arg{\vect{r}} \cdot \field{E}^\text{inc} d\vect{r}$
\item $Z_{mn}^{HJ} = \int_{D_m} \frac{1}{2} \vect{g}_m \arg{\vect{r}} \cdot \vect{\hat{n}} \times \current{f}_{n}\arg{\vect{r}}d\vect{r} + \int_{D_m} \vect{g}_m \arg{\vect{r}} \cdot \left(\int_{D_n} \nabla G_1\left(\vect{r}, \vect{r}'\right) \times \current{f}_n\left(\vect{r}'\right) d\vect{r}'\right) d\vect{r}$
\item $V_m^{H} = -\int_{D_m}\vect{g}_m \arg{\vect{r}} \cdot \field{H}^\text{inc} d\vect{r}$.
\end{itemize}


\chapter{Method of moments solution terms}
\label{annex:MoM}

%
\par
In this section we are going to detail how to compute the terms of the MoM matrices that appear in (\ref{eqn:EFIE-O_PEC_2}) and (\ref{eqn:MFIE-O_PEC_2}) for PEC targets, namely
\begin{gather} 
D_{mn} = \int_{D_m}\vect{g}_m \arg{\vect{r}} \cdot \left( \left(\nabla \nabla \cdot + k^2\right) \int_{D_n} G\left(\vect{r}, \vect{r}'\right) \current{f}_n\left(\vect{r}'\right) d\vect{r}' \right) d\vect{r} \label{eqn:MoM terms I_1}\\
K_{mn} = \int_{D_m} \vect{g}_m \arg{\vect{r}} \cdot \left(\int_{D_n} \nabla G\left(\vect{r}, \vect{r}'\right) \times \current{f}_n\left(\vect{r}'\right) d\vect{r}'\right) d\vect{r} \label{eqn:MoM terms I_2}\\
J_{mn} = \int_{D_m} \frac{1}{2} \vect{g}_m \arg{\vect{r}} \cdot \vect{\hat{n}} \times \current{f}_{n}\arg{\vect{r}}d\vect{r}. \label{eqn:MoM terms I_3}
\end{gather}
As usual, the homogeneous space Green's function $G = \frac{e^{-jk\left|\vect{r}-\vect{r}' \right|}}{4 \pi \left|\vect{r}-\vect{r}' \right|}$, and $k = \omega \sqrt{\varepsilon \mu} = \omega \sqrt{\varepsilon_0 \varepsilon_r \mu_0 \mu_r}$.

\section{Computation of $D_{mn}$}
%
\par
First let us take care of the derivative term. It is written as:
\begin{equation}
\nabla \nabla \cdot \int_{D_n} G\arg{\vect{r},\vect{r}'} \vect{f}_n\arg{\vect{r}'} \; \ud S',
\end{equation}
where the integration is performed on the domain $D_n$ of the RWG basis function $\vect{f}_n$. By using the fact that $\nabla \cdot G = -\nabla' \cdot G$ and basic vector identities, we can write the following sequence of equalities:
\begin{equation*}
\begin{split}
\nabla \nabla \cdot \int_{D_n} G \vect{f}_n \; \ud S' &= \nabla \int_{D_n} \nabla \cdot \left(G \vect{f}_n \right) \; \ud S' \\
&= \nabla \int_{D_n} \nabla G \cdot \vect{f}_n \; \ud S' \\
&= -\nabla \int_{D_n} \nabla' G \cdot \vect{f}_n \; \ud S' \\
&= -\nabla \int_{D_n} \left[\nabla' \cdot \left(G \vect{f}_n \right) - G \nabla_S^\prime \cdot \vect{f}_n\right]\; \ud S'\\
&= -\nabla \oint_{\partial D_n} \uvect{m}_n \cdot G \vect{f}_n \; \ud l + \nabla \int_{D_n} G \nabla_S^\prime \cdot \vect{f}_n  \; \ud S'
\end{split}
\end{equation*}
where $\partial D_n$ denotes integration on the border of $D_n$. Due to the properties of the RWG basis functions given at section \ref{subsubsec:Basis functions: definition and properties}, the contour integral yields zero. We can therefore write that 
\begin{equation}
\nabla \nabla \cdot \int_{D_n} G \vect{f}_n \; \ud S' = \nabla \int_{D_n} G \nabla_S^\prime \cdot \vect{f}_n  \; \ud S',
\end{equation}
which allows us to decompose $D_{mn}$ as follows:
\begin{equation}
D_{mn} = \underbrace{\int_{D_m} \vect{g}_m\arg{\vect{r}} \cdot \left(\nabla \int_{D_n} G\arg{\vect{r},\vect{r}'} \; \nabla_S^\prime \cdot \vect{f}_n  \; d \vect{r}'\right) \ud \vect{r} }_{\triangleq D_{mn,1}} + k^2\underbrace{\int_{D_m}\vect{g}_m \arg{\vect{r}} \cdot \left( \int_{D_n} G\arg{\vect{r},\vect{r}'} \vect{f}_n\arg{\vect{r}}\right) d\vect{r}}_{\triangleq D_{mn,2}}.
\end{equation}


\subsection{Testing with $\vect{g}_m = \vect{f}_m$: $D_{mn}^t$}
\subsubsection{$D_{mn,1}^t$}
%
\par
First we modify integral $D_{mn,1}^t$. The gradient applied to the inside inner product can be applied on the testing function by integrating by parts and by applying Gauss' divergence theorem \cite{Rao_82, Taskinen_03}:
\begin{equation} \label{eqn:Gauss' divergence theorem}
\begin{split}
\int_{D_m} \vect{f}_m \cdot \nabla \phi \; \ud S &= \int_{D_m} \nabla \cdot \left( \vect{f}_m \phi \right) \; \ud S - \int_{D_m} \phi \nabla_S \cdot \vect{f}_m \; \ud S \\
&= \int_{\partial D_m} \left(\vect{f}_m \phi \right) \cdot \uvect{m} \; \ud l - \int_{D_m} \phi \nabla_S \cdot \vect{f}_m \; \ud S
\end{split}
\end{equation}
where $\phi$ is a scalar function, $\partial D_m$ is the contour of $D_m$ and $\uvect{m} = \uvect{l} \times \uvect{n}$ is the normal to the contour of the basis function domain. The contour integral yields zero, due to the fact that, on the edges of the RWG basis function domain $D_m$, $\vect{f}_m$ is parallel to $\uvect{l}_m$. We therefore have that
\begin{equation}
D_{mn,1}^t = -\int_{D_m}  \nabla_S \cdot \vect{f}_m\arg{\vect{r}}  \left( \int_{D_n} G\arg{\vect{r},\vect{r}'} \; \nabla_S^\prime \cdot \vect{f}_n  \; d \vect{r}'\right) \ud \vect{r}.
\end{equation}
%
\par
From the divergence properties of the RWG basis functions given by (\ref{eqn:RWG divergence}), it is easy to see that $D_{mn,1}^t$ is constituted by the following sum:
\begin{equation} \label{eqn:MoM scalar potential term}
D_{mn,1}^t = \sum_{p, \, T_m^p \in D_m} \; \sum_{q, \, T_n^q \in D_n} -4 C_m^p C_n^q \int_{T_m^p} \int_{T_n^q} G \; \ud S' \ud S
\end{equation}
with $C_m^p = \frac{S_m^p l_m}{2 A_m^p}$, where $S_m^p$ is the sign of test function $m$ in triangle $p$, $l_m$ is the length of edge $m$, and $A_m^p$ is the area of triangle $T_m^p$. The sums are performed on $p$ and $q$ for which $T_m^p$ and $T_n^q$ are the triangles that form half of $D_m$ and $D_n$ respectively. We can note that the term under the integration sign is independent upon the basis function. This constatation leads us to remark that, instead of evaluating the MoM integrals RWG-wise, we will perform them triangle-wise, and contribute back the terms into the MoM matrix with the appropriate coefficients for all basis and test functions that pertain to the two triangles \cite{Rao_82}. Since for closed surfaces, there are three basis functions per triangle, and because the majority of the time is spent on these integrals, performing them triangle-wise will save a non-negligible amount of computation time. In the remainder of the text, the two summation symbols will be dropped for clarity.
%
\par
One more comment must be made about (\ref{eqn:MoM scalar potential term}). $G$ is singular when $\vect{r} = \vect{r}'$, and this renders the numerical integration very imprecise if $T_m^p$ is close to $T_n^q$. This singularity must be properly extracted. This is done thoroughly in \cite{Taskinen_03} and will not be discussed here.

\subsubsection{$D_{mn,2}^t$}
%
\par
Now let us develop $D_{mn,2}^t$. From definition (\ref{eqn:RWG definition}) of the RWG basis functions, it is immediate to see that $D_{mn,2}^t$ is constituted by combinations of the following term:
\begin{multline}  \label{eqn:MoM vector potential term}
C_m^p C_n^q \int_{T_m^p} \left(\vect{r}-\vect{r}_m^p \right) \cdot \int_{T_n^q} G \left(\vect{r}'-\vect{r}_n^q \right) \ud S' \ud S = C_m^p C_n^q \Bigg[ \int_{T_m^p} \vect{r} \cdot \int_{T_n^q} G \; \vect{r}' \ud S' \ud S  - \vect{r}_n^q \cdot \int_{T_m^p} \vect{r} \int_{T_n^q} G \; \ud S' \ud S \\
- \vect{r}_m^p \cdot \int_{T_m^p} \int_{T_n^q} G \; \vect{r}' \ud S' \ud S + \vect{r}_m^p \cdot \vect{r}_n^q \int_{T_m^p} \int_{T_n^q} G \; \ud S' \ud S \Bigg]
\end{multline}
where $\vect{r}_m^p$ is the vector position of the node belonging to triangle $p$ and opposite to edge $m$. The four integrals on the right-hand side of the above equation are independent of the edges, and each can again be performed triangle-wise. Only their recombination is edge-dependent.

\subsection{Testing with $\vect{g}_m = \uvect{n} \times \vect{f}_m$: $D_{mn}^n$}

\subsubsection{$D_{mn,1}^n$}
%
\par
Rewriting explicitly $D_{mn,1}^n$ with $\vect{g}_m = \uvect{n} \times \vect{f}_m$, we can see that it will be a combination of the following term:
\begin{equation}
2 C_m^p C_n^q \int_{T_m^p} \big(\uvect{n}_m^p \times \left(\vect{r}-\vect{r}_m^p \right) \big) \cdot \int_{T_n^q} \nabla G \; \ud S' \ud S
\end{equation}
This time we cannot move the gradient onto the testing function in $D_{mn,1}^n$, because $\uvect{n} \times \vect{f}_m$ is not continuous on the triangle pair that forms $D_m$ \cite{Taskinen_03}, as the triangles normals differ. The integral may be further decomposed as:
\begin{equation} \label{eqn:n_x_fm_grad_G_terms}
\begin{split}
\int_{T_m^p} \big(\uvect{n}_m^p \times \left(\vect{r}-\vect{r}_m^p \right) \big) \cdot \int_{T_n^q} \nabla G \; \ud S' \ud S & = \int_{T_m^p} \left(\uvect{n}_m^p \times \vect{r} \right) \cdot \int_{T_n^q} \nabla G \; \ud S' \ud S - \left(\uvect{n}_m^p \times \vect{r}_m^p\right) \cdot \int_{T_m^p} \int_{T_n^q} \nabla G \; \ud S' \ud S \\
& = \uvect{n}_m^p \cdot \left(\int_{T_m^p} \vect{r} \times \int_{T_n^q} \nabla G \; \ud S' \ud S - \vect{r}_m^p \times \int_{T_m^p} \int_{T_n^q} \nabla G \; \ud S' \ud S\right).
\end{split}
\end{equation}
%
\par
The kernel involved in the terms contained in (\ref{eqn:n_x_fm_grad_G_terms}) is highly singular, as it involves an integral of a $1/R^2$ singularity, in contrast with an integral of a $1/R$ singularity usually involved with $G$ \cite{Taskinen_03}. When domains of basis functions $m$ and $n$ are ``sufficiently far away'' from each other, a regular numerical integration of the terms appearing in (\ref{eqn:n_x_fm_grad_G_terms}) should not cause any trouble. But if the test and basis function overlap, we will not be able to extract analytically the singularity, and a numerical method will have to evaluate a great number of times the integrand for obtaining a precise value of the integral. 
%
\par
However, an elegant transformation may be applied to the integrand by noting that
\begin{equation}
\begin{split}
\int_{T_m^p} \big(\uvect{n}_m^p \times \left(\vect{r}-\vect{r}_m^p \right) \big) \cdot \int_{T_n^q} \nabla G \; \ud S' \ud S &= \uvect{n}_m^p \cdot \int_{T_m^p} \left(\vect{r}-\vect{r}_m^p \right) \times \int_{T_n^q} \nabla G \; \ud S' \ud S \\
&= - \uvect{n}_m^p \cdot \left[ \int_{T_m^p} \int_{T_n^q} \nabla \times \big( G \left( \vect{r}-\vect{r}_m^p \right)\big) \; \ud S' \ud S - \int_{T_m^p} \int_{T_n^q} G \nabla \times \left( \vect{r}-\vect{r}_m^p \right) \; \ud S' \ud S\right] \\
&=  - \uvect{n}_m^p \cdot \int_{\partial T_m^p} \big(\uvect{m}_m^p \times  \left( \vect{r}-\vect{r}_m^p \right) \big) \int_{T_n^q}  G \; \ud S' \ud l 
\end{split}
\end{equation}
where use of identity $\nabla \times \left(a \vect{b}\right) = a \nabla \times \vect{b} - \vect{b} \times \nabla a$ has been made. The last equality is due to the fact that the rotational of the position vector $ \left( \vect{r}-\vect{r}_m^p \right)$ is zero. By noting that $\uvect{m} = \uvect{l} \times \uvect{n}$, we finally obtain
\begin{equation}
\begin{split}
\int_{T_m^p} \big(\uvect{n}_m^p  \times \left(\vect{r}-\vect{r}_m^p \right) \big) \cdot \int_{T_n^q} \nabla G \; \ud S' \ud S &= \int_{\partial T_m^p} -\uvect{l}_m^p \cdot  \left( \vect{r}-\vect{r}_m^p \right)  \int_{T_n^q}  G \; \ud S' \ud l \\
& = \int_{\partial T_m^p} -\uvect{l}_m^p \cdot \vect{r}  \int_{T_n^q}  G \; \ud S' \ud l + \vect{r}_m^p \cdot \int_{\partial T_m^p}\uvect{l}_m^p  \int_{T_n^q}  G \; \ud S' \ud l
\end{split}
\end{equation}
which is a more elegant form than its counterpart (23) of \cite{Taskinen_03}. Singularity $\frac{1}{R^2}$ has been reduced to $\frac{1}{R}$ and can therefore be analytically extracted even for overlapping basis and test functions. 

\subsubsection{$D_{mn,2}^n$}
%
\par
Let us now develop $D_{mn,2}^n$. We will have the following terms:
\begin{multline}
C_m^p C_n^q \int_{T_m^p} \big( \uvect{n}_m^p \times \left(\vect{r} - \vect{r}_m^p \right)\big) \cdot \int_{T_n^q} G \left(\vect{r}'-\vect{r}_n^q \right) \; \ud S' \ud S \\
= C_m^p C_n^q  \Bigg[\int_{T_m^p} \left(\uvect{n}_m^p \times \vect{r} \right) \cdot \int_{T_n^q} G \; \vect{r}' \; \ud S' \ud S - \vect{r}_n^q \cdot \left(\uvect{n} \times \int_{T_m^p} \vect{r} \int_{T_n^q} G \; \ud S' \ud S \right)  \\
- \left(\uvect{n}_m^p \times \vect{r}_m^p\right)  \cdot \int_{T_m^p} \int_{T_n^q} G \; \vect{r}'\; \ud S' \ud S + \vect{r}_n^q \cdot \left(\uvect{n}_m^p \times \vect{r}_m^p\right) \int_{T_m^p} \int_{T_n^q} G \; \ud S' \ud S \Bigg].
\end{multline}
These terms do not pose any particular problems, and some of them are already present in (\ref{eqn:MoM vector potential term}).

\section{Computation of $K_{mn}$}
%
\par
Let us rewrite $K_{mn} = \int_{D_m} \vect{g}_m \cdot \int_{D_n} \nabla G \times \vect{f}_n \; \ud S' \ud S$. Note that, if the two triangles $T_m^p$ and $T_n^q$ that form respectively half of $D_m$ and half of $D_n$ are coplanar, the corresponding contribution $K_{mn}^{pq} = 0$, since in that case $\nabla G$ is contained in the same plane as $\vect{f}_n$ and $\vect{g}_m$, and therefore $\nabla G \times \vect{f}_n$ and $\vect{g}_m$ are perpendicular. \textit{A fortiori}, $K_{mn}^{pq}$ will be zero if $T_m^p = T_n^q$.

\subsection{Testing with $\vect{g}_m = \vect{f}_m$: $K_{mn}^t$}
%
\par
After replacing $\left(\vect{r}' - \vect{r}_n^q\right)$ by $\left(\vect{r}' - \vect{r}\right) + \left(\vect{r} - \vect{r}_m^p\right) + \left(\vect{r}_m^p - \vect{r}_n^q\right)$ and performing a few manipulations, it can be shown that (\ref{eqn:MoM terms I_2}) implies terms of type (see (20) of \cite{Taskinen_03}):
\begin{multline} \label{eqn:G_HJ_fs_testing_decomposition}
C_m^p C_n^q \int_{T_m^p} \left(\vect{r} - \vect{r}_m^p \right) \cdot \left[-\left(\vect{r}_m^p - \vect{r}_n^q\right) \times \int_{T_n^q} \nabla G \; \ud S'\right] \; \ud S \\
= C_m^p C_n^q \left(\vect{r}_m^p - \vect{r}_n^q\right) \cdot \left[\int_{T_m^p} \vect{r} \times \int_{T_n^q} \nabla G \; \ud S' \ud S - \vect{r}_m^p \times \int_{T_m^p} \int_{T_n^q}\nabla G \; \ud S' \ud S \right].
\end{multline}

\subsection{Testing with $\vect{g}_m = \uvect{n} \times \vect{f}_m$: $K_{mn}^n$}
%
\par
In this case, by replacing $\left(\vect{r}'-\vect{r}_n^q \right)$ by $\left(\vect{r}'-\vect{r} \right) + \left(\vect{r}-\vect{r}_n^q \right)$, (\ref{eqn:MoM terms I_2}) implies terms of type:
\begin{multline} \label{eqn:G_HJ_fs_n_x_RWG_testing_decomposition}
C_m^p C_n^q \int_{T_m^p} \big( \uvect{n}_m^p \times \left(\vect{r} - \vect{r}_m^p \right) \big) \cdot \left[-\left(\vect{r} - \vect{r}_n^q\right) \times \int_{T_n^q} \nabla G \; \ud S'\right] \; \ud S \\ 
= -C_m^p C_n^q \left\{ \int_{T_m^p} \left(\uvect{n}_m^p \times \vect{r}\right) \cdot \left[\vect{r} \times \int_{T_n^q} \nabla G \; \ud S' \right] \ud S \right. \\
+ \vect{r}_n^q \cdot \int_{T_m^p} \left(\uvect{n}_m^p \times \vect{r}\right) \times \int_{T_n^q} \nabla G \; \ud S' \ud S  - \left(\uvect{n}_m^p\times\vect{r}_m^p \right) \cdot \int_{T_m^p} \vect{r} \times \int_{T_n^q} \nabla G \; \ud S' \ud S \\
\left. - \vect{r}_n^q \cdot \left[\left(\uvect{n} \times \vect{r}_m^p \right) \times \int_{T_m^p} \int_{T_n^q}\nabla G \; \ud S' \ud S\right]\right\}.
\end{multline}
Note that two terms of (\ref{eqn:G_HJ_fs_n_x_RWG_testing_decomposition}) are present in (\ref{eqn:G_HJ_fs_testing_decomposition}).

\section{Computation of $J_{mn}$}

\subsection{Testing with $\vect{g}_m = \vect{f}_m$: $J_{mn}^t$}
%
\par
$J_{mn}$ is nonzero only for overlapping triangles $p$ and $q$, and will involve combinations of terms of the type
\begin{equation} \label{eqn:MMPIE_J_RWG_testing_diag_term}
C_m^p C_n^p \int_{T_m^p} \left(\vect{r}-\vect{r}_m^p\right) \cdot \left(\uvect{n}_m^p \times \left(\vect{r}-\vect{r}_n^p\right) \right) \; \ud S =  - C_m^p C_n^p \uvect{n}_m^p \cdot \left[\left(\vect{r}_n^p-\vect{r}_m^p\right) \times \int_{T_m^p} \vect{r} \; \ud S + \left(\vect{r}_m^p \times \vect{r}_n^p\right) \int_{T_m^p} 1 \; \ud S \right].
\end{equation}

\subsection{Testing with $\vect{g}_m = \uvect{n} \times \vect{f}_m$: $J_{mn}^n$}
%
\par
We can immediately write that
\begin{equation}
\innerprod{\uvect{n} \times \vect{f}_m}{;}{\uvect{n} \times \vect{f}_n}{D_m} = \innerprod{\vect{f}_m}{;}{\vect{f}_n}{D_m} 
\end{equation}
where the right-hand term can be further decomposed using basic vector formulas as
\begin{equation} \label{eqn:MMPIE_J_n_x_RWG_testing_diag_term}
C_m^p C_n^p \int_{T_m^p} \left(\vect{r}-\vect{r}_m^p \right) \cdot \left(\vect{r}-\vect{r}_n^p\right) \; \ud S = C_m^p C_n^p \left[ \int_{T_m^p} \left|\vect{r}\right|^2 \; \ud S - \left(\vect{r}_m^p + \vect{r}_n^p \right) \cdot \int_{T_m^p} \vect{r} \; \ud S + \left(\vect{r}_m^p \cdot \vect{r}_n^p \right) \int_{T_m^p} 1 \; \ud S\right].
\end{equation}


\section{Computation of the MoM excitation vectors}
\label{app:Computation of the MoM excitation vectors}
%
\par
The nonzero terms involved in the computation of the excitation vectors $\Array{V}{}{E}$ and $\Array{V}{}{H}$ have the following generic form:
\begin{equation}
V_m^{P} = -\int_{D_m}\vect{g}_m \arg{\vect{r}} \cdot \field{P}^\text{inc} d\vect{r}
\end{equation}
where $\vect{P}$ can be $\vect{E}$ or $\vect{H}$. If tested with $\vect{g}_m = \vect{f}_m$, we immediately have that
\begin{equation}
-\int_{D_m}\vect{f}_m \arg{\vect{r}} \cdot \field{P}^\text{inc} d\vect{r} = -\frac{l_m}{2 A_m^+}\int_{T_m^+} \left(\vect{r} - \vect{r}_m^+ \right) \cdot \vect{P}^\text{inc} \; \ud S + \frac{l_m}{2 A_m^-}\int_{T_m^-} \left(\vect{r} - \vect{r}_m^- \right) \cdot \vect{P}^\text{inc} \; \ud S.
\end{equation}
If tested with $\vect{g}_m = \uvect{n} \times \vect{f}_m$, we have 
\begin{equation}
-\int_{D_m} \left(\uvect{n}\times\vect{f}_m\arg{\vect{r}}\right)  \cdot \field{P}^\text{inc} d\vect{r} = -\frac{l_m}{2 A_m^+}\int_{T_m^+} \big( \uvect{n}_m^+ \times \left(\vect{r} - \vect{r}_m^+ \right) \big) \cdot \vect{P}^\text{inc} \; \ud S + \frac{l_m}{2 A_m^-}\int_{T_m^-} \big( \uvect{n}_m^- \times \left(\vect{r} - \vect{r}_m^- \right)\big) \cdot \vect{P}^\text{inc} \; \ud S.
\end{equation}



\chapter{MLFMA interactions}
%
\par
The method of moments EFIE (\ref{eqn:EFIE-O_PEC_2}) and MFIE (\ref{eqn:MFIE-O_PEC_2}) involve the calculation of the following interactions:
\begin{align}
& \int_{D_p} \current{g}_p\left(\vect{r}\right) \cdot \left(\left(\nabla \nabla \cdot + k_i^2\right) \int_{D_q} G_i\left(\vect{r}, \vect{r}'\right) \current{f}_q\left(\vect{r}'\right) d\vect{r}' \right) d\vect{r} \qquad \text{(D-type)} \label{eqn:interaction_D}\\
& \int_{D_p} \current{g}_p\left(\vect{r}\right) \cdot \left( \int_{D_q} \nabla G_i\left(\vect{r}, \vect{r}'\right) \times \current{f}_q\left(\vect{r}'\right) d\vect{r}' \right) d\vect{r} \qquad \text{(K-type)} \label{eqn:interaction_K}
\end{align}
with $\current{g}_p = \current{f}_p$ or $\current{g}_p = \uvect{n}\times\current{f}_p$, and with
\begin{equation*}
G_i(\vect{r}, \vect{r}') = \frac{e^{-jk_i \left|\vect{r} - \vect{r}'  \right|}}{4 \pi \left|\vect{r} - \vect{r}'  \right|}.
\end{equation*}
We will now derive the MLFMA expressions for the far field interactions \cite{Chew_01}. First, the addition theorem states that:
\begin{equation}\label{eqn:addition_theorem}
\frac{e^{-jk \left|\vect{D} + \vect{d}  \right|}}{4 \pi \left|\vect{D} + \vect{d} \right|} \simeq \frac{-jk}{4 \pi} \sum_{l = 0}^L \left(-1\right)^l \left(2l+1 \right) j_l \! \left(k d\right) h_l^{(2)}\! \left(k D\right) P_l\left(\vect{\hat{d}} \cdot \vect{\hat{D}} \right)
\end{equation}
where $j_l$, $h_l$ and $P_l$ are the spherical Bessel function, spherical Hankel function and Legendre polynomial. Another elementary identity reads
\begin{equation}
\int d^2\vect{\hat{k}} e^{-j\vect{k} \cdot \vect{d}} P_l \! \left(\vect{\hat{k}} \cdot \vect{\hat{D}}\right) = 4 \pi \left(-j \right)^l j_l \! \left(k d\right) P_l \! \left(\vect{\hat{d}} \cdot \vect{\hat{D}}\right)
\end{equation}
which can be plugged right back into (\ref{eqn:addition_theorem}) to yield
\begin{equation}\label{eqn:addition_theorem_2}
\begin{split}
\frac{e^{-jk \left|\vect{D} + \vect{d}  \right|}}{4 \pi \left|\vect{D} + \vect{d} \right|} &\simeq \frac{-jk}{16 \pi^2} \int d^2\vect{\hat{k}} e^{-j\vect{k} \cdot \vect{d}} \sum_{l = 0}^L \left(\frac{-1}{-j}\right)^l \left(2l+1 \right) h_l^{(2)}\! \left(k D\right) P_l\left(\vect{\hat{k}} \cdot \vect{\hat{D}} \right) \\
&\simeq \frac{-jk}{16 \pi^2} \int d^2\vect{\hat{k}} e^{-j\vect{k} \cdot \vect{d}} T_L\left(\vect{k}, \vect{D} \right)
\end{split}
\end{equation}
with
\begin{equation}
T_L\left(\vect{k}, \vect{D} \right) = \sum_{l = 0}^L \left(-j\right)^l \left(2l+1 \right) h_l^{(2)}\! \left(k D\right) P_l\left(\vect{\hat{k}} \cdot \vect{\hat{D}} \right).
\end{equation}
Now, if we write 
\begin{equation}\label{eqn:r_rprime}
\vect{r} - \vect{r}' = \underbrace{\vect{r} - \vect{r}_m}_{\vect{R}_{m}} + \underbrace{\vect{r}_m - \vect{r}_{m'}}_{\vect{R}_{mm'}} + \underbrace{\vect{r}_{m'} - \vect{r}'}_{-\vect{R}_{m'}}
\end{equation}
by letting $\vect{D} = \vect{R}_{mm'}$ and $\vect{d} = \vect{R}_{m} - \vect{R}_{m'}$, we can rewrite (\ref{eqn:addition_theorem_2}):
\begin{equation}\label{eqn:addition_theorem_3}
\frac{e^{-jk \left|\vect{r} - \vect{r}'  \right|}}{4 \pi \left|\vect{r} - \vect{r}' \right|} = \int d^2\vect{\hat{k}} e^{-j\vect{k} \cdot \left( \vect{R}_{m} - \vect{R}_{m'} \right)} \alpha_{mm'}\left(\vect{k}, \vect{R}_{mm'} \right)
\end{equation}
where
\begin{equation}
\alpha_{mm'}\left(\vect{k}, \vect{R}_{mm'} \right) = \frac{-jk}{16 \pi^2} \sum_{l = 0}^L \left(-j\right)^l \left(2l+1 \right) h_l^{(2)}\! \left(k R_{mm'}\right) P_l\left(\vect{\hat{k}} \cdot \vect{\hat{R}}_{mm'} \right).
\end{equation}

\section{D-type interaction}
%
\par
First, it is to be noted that 
\begin{equation}
\left(\nabla \nabla \cdot + k^2\right) \int_{D_q} G\left(\vect{r}, \vect{r}'\right) \current{f}_q\left(\vect{r}'\right) d\vect{r}' = \int_{D_q} \left(\nabla \nabla G\left(\vect{r}, \vect{r}'\right) \right) \cdot \current{f}_q\left(\vect{r}'\right) d\vect{r}' + k^2 \int_{D_q} G\left(\vect{r}, \vect{r}'\right) \current{f}_q\left(\vect{r}'\right) d\vect{r}'.
\end{equation}
The double derivation is applied only to $G$, as $\current{f}_n$ is a function of $\vect{r}'$. Using (\ref{eqn:addition_theorem_3}), we can write 
\begin{equation}
\nabla G\left(\vect{r}, \vect{r}'\right) = -jk \int \vect{\hat{k}} e^{-j\vect{k} \cdot \left( \vect{R}_{m} - \vect{R}_{m'} \right)}  \alpha_{mm'} \! \left(\vect{k}, \vect{R}_{mm'} \right) d^2\vect{\hat{k}}
\end{equation}
and the double derivation results in
\begin{equation}\label{eqn:double_deriv}
\nabla \nabla G\left(\vect{r}, \vect{r}'\right) = -k^2  \int \vect{\hat{k}}\vect{\hat{k}} e^{-j\vect{k} \cdot \left( \vect{R}_{m} - \vect{R}_{m'} \right)}  \alpha_{mm'} \! \left(\vect{k}, \vect{R}_{mm'} \right) d^2\vect{\hat{k}}.
\end{equation}
%
\par
Let us rewrite (\ref{eqn:interaction_D}) by using (\ref{eqn:addition_theorem_3}) and (\ref{eqn:double_deriv}):
\begin{equation}
\begin{split}
\int_{D_p} \current{g}_p\left(\vect{r}\right) \cdot \left(\left(\nabla \nabla \cdot + k^2\right) \int_{D_q} G\left(\vect{r}, \vect{r}'\right) \current{f}_q\left(\vect{r}'\right) d\vect{r}' \right) d\vect{r} &\simeq k^2 \int_{D_p}d \vect{r} \int d^2\vect{\hat{k}} e^{-j\vect{k} \cdot \vect{R}_{m}}\current{g}_p\left(\vect{r}\right) \cdot \\ & \qquad \alpha_{mm'}\left(\vect{k}, \vect{R}_{mm'}\right) \int_{D_q} \left( \overline{\overline{I}}-\vect{\hat{k}}\vect{\hat{k}} \right) \cdot e^{j\vect{k} \cdot \vect{R}_{m'}} \vect{f}_q\left(\vect{r}'\right) d \vect{r}' \\
&\simeq k^2 \int d^2\vect{\hat{k}} \; \vect{V}_{mp} \! \left(\vect{\hat{k}}\right) \cdot \alpha_{mm'}\left(\vect{k}, \vect{R}_{mm'} \right) \vect{V}_{m'q} \! \left(\vect{\hat{k}}\right).
\end{split}
\end{equation}
We can now clearly see the three MLFMA terms:
\begin{itemize}
\item aggregation: $\vect{V}_{m'q} \! \left(\vect{\hat{k}}\right) = \int_{D_q} \left( \overline{\overline{I}}-\vect{\hat{k}}\vect{\hat{k}} \right) \cdot e^{j\vect{k} \cdot \vect{R}_{m'}} \vect{f}_q\left(\vect{r}'\right) d \vect{r}'$
\item translation: $\alpha_{mm'}\left(\vect{k}, \vect{R}_{mm'} \right)$
\item disaggregation: $\vect{V}_{mp} \! \left(\vect{\hat{k}}\right) = \int_{D_p} \left( \overline{\overline{I}}-\vect{\hat{k}}\vect{\hat{k}} \right) \cdot e^{-j\vect{k} \cdot \vect{R}_{m}} \vect{g}_p\left(\vect{r}\right) d \vect{r}$
\end{itemize}
where the supplementary $\left(\overline{\overline{I}}-\vect{\hat{k}}\vect{\hat{k}} \right)$ in $\vect{V}_{mp} \! \left(\vect{\hat{k}}\right)$ comes from the fact that $\left(\overline{\overline{I}}-\vect{\hat{k}}\vect{\hat{k}} \right) = \left(\overline{\overline{I}}-\vect{\hat{k}}\vect{\hat{k}} \right) \cdot \left(\overline{\overline{I}}-\vect{\hat{k}}\vect{\hat{k}} \right)$.



\bibliography{references}
\end{document}

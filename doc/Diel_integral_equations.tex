\chapter{MoM solution of the integral equations for dielectrics}

\section{MoM matrix equations}
%
\par
We will now examine the Method-of-Moments solution for homogeneous dielectric bodies. Let's first recall the integral equations:
\begin{equation*}
\boxed{
\field{H}^\text{inc} = - \frac{1}{2} \vect{\hat{n}} \times \current{J}_{S,1} - \operator{K}_1 \left(\current{J}_{S,1}\right) - \frac{1}{j \omega \mu_1} \operator{D}_1 \left( \current{M}_{S,1}\right) 
} \qquad MFIE-O \tag{\ref{eqn:MFIE-O} recalled}
\end{equation*}
\begin{equation*}
\boxed{
\field{E}^\text{inc} =   - \frac{1}{j \omega \varepsilon_1} \operator{D}_1 \left( \current{J}_{S,1}\right) + \frac{1}{2} \vect{\hat{n}} \times \current{M}_{S,1} + \operator{K}_1 \left(\current{M}_{S,1}\right)
} \qquad EFIE-O \tag{\ref{eqn:EFIE-O} recalled}
\end{equation*}
\begin{equation*}
\boxed{
0 = - \frac{1}{2} \vect{\hat{n}} \times \current{J}_{S,1} + \operator{K}_2\left( \current{J}_{S,1}\right) + \frac{1}{j \omega \mu_2} \operator{D}_2\left(\current{M}_{S,1}\right)
} \qquad MFIE-I \tag{\ref{eqn:MFIE-I} recalled}
\end{equation*}
\begin{equation*}
\boxed{
0 = \frac{1}{j \omega \varepsilon_2} \operator{D}_2\left(\current{J}_{S,1}\right) + \frac{1}{2} \vect{\hat{n}} \times \current{M}_{S,1} -  \operator{K}_2\left( \current{M}_{S,1}\right)
} \qquad EFIE-I \tag{\ref{eqn:EFIE-I} recalled}
\end{equation*}
with
\begin{align*}
\operator{D}_i\left(\current{X}\right) &= \left(\nabla \nabla \cdot + k_i^2\right) \int_{S} G_i\left(\vect{r}, \vect{r}'\right) \current{X}\left(\vect{r}'\right) d\vect{r}' \\
\operator{K}_i\left(\current{X}\right) &= \int_S \nabla G_i\left(\vect{r}, \vect{r}'\right) \times \current{X}\left(\vect{r}'\right) d\vect{r}'.
\end{align*}
%
\par
As in section \ref{subsubsec:Basis functions: definition and properties}, the two unknown currents $\vect{J}_{S, 1}$ and $\vect{M}_{S, 1}$ are decomposed in a RWG basis with their respective coefficients, as in (\ref{eqn:current_decomposition}), and we therefore have for the currents 
\begin{equation}\label{eqn:J_M_current_decomposition}
\vect{J}\arg{\vect{r}} \simeq \sum_{n = 1}^{N} I_n \vect{f}_n \arg{\vect{r}}, \qquad \vect{M}\arg{\vect{r}} \simeq \sum_{n = 1}^{N} M_n \vect{f}_n \arg{\vect{r}}.
\end{equation}
With (\ref{eqn:J_M_current_decomposition}), the discretized integral equations become:
\begin{equation}
-\field{H}^\text{inc} =  \frac{1}{2} \sum_{n=1}^N I_n \vect{\hat{n}} \times \current{f}_n\arg{\vect{r}} + \sum_{n=1}^{N} I_n \operator{K}_1\left(\current{f}_n\right) + \frac{1}{j \omega \mu_1} \sum_{n=1}^{N} M_n \operator{D}_1\left(\current{f}_n\right)  \qquad MFIE-O 
\end{equation}
\begin{equation}
-\field{E}^\text{inc} = \frac{1}{j \omega \varepsilon_1} \sum_{n=1}^{N} I_n \operator{D}_1\left(\current{f}_n\right) -\frac{1}{2} \sum_{n=1}^N M_n \vect{\hat{n}} \times \current{f}_n\arg{\vect{r}}  -  \sum_{n=1}^{N} M_n \operator{K}_1\left(\current{f}_n\right) \qquad EFIE-O 
\end{equation}
\begin{equation}
0 =  -\frac{1}{2} \sum_{n=1}^N I_n \vect{\hat{n}} \times \current{f}_n\arg{\vect{r}} + \sum_{n=1}^{N} I_n \operator{K}_2\left(\current{f}_n\right) + \frac{1}{j \omega \mu_2} \sum_{n=1}^{N} M_n \operator{D}_2\left(\current{f}_n\right)  \qquad MFIE-I 
\end{equation}
\begin{equation}
0 =   \frac{1}{j \omega \varepsilon_2} \sum_{n=1}^{N} I_n \operator{D}_2\left(\current{f}_n\right) + \frac{1}{2} \sum_{n=1}^N M_n \vect{\hat{n}} \times \current{f}_n\arg{\vect{r}} -  \sum_{n=1}^{N} M_n \operator{K}_2\left(\current{f}_n\right) \qquad EFIE-I 
\end{equation}
with
\begin{align*}
\operator{D}_i\left(\current{f}_n\right) &= \left(\nabla \nabla \cdot + k_i^2\right) \int_{D_n} G_i\left(\vect{r}, \vect{r}'\right) \current{f}_n\left(\vect{r}'\right) d\vect{r}' \\
\operator{K}_i\left(\current{f}_n\right) &= \int_{D_n} \nabla G_i\left(\vect{r}, \vect{r}'\right) \times \current{f}_n\left(\vect{r}'\right) d\vect{r}'
\end{align*}
and where $D_n$ is the domain of basis function $\vect{f}_n$.
%
\par
Finally, testing with $\vect{g}_m \arg{\vect{r}}$ (\ref{eqn:testing_function}), the method of moments expressions for the integral equations become:
\begin{equation}\label{eqn:MoM MFIE-O}
\boxed{V_m^H =  \sum_{n=1}^N I_n J_{mn} + \sum_{n=1}^{N} I_n K_{mn}^{(1)} + \frac{1}{j \omega \mu_1} \sum_{n=1}^{N} M_n D_{mn}^{(1)}} \qquad MFIE-O 
\end{equation}
\begin{equation}\label{eqn:MoM EFIE-O}
\boxed{V_m^E =  \frac{1}{j \omega \varepsilon_1} \sum_{n=1}^{N} I_n D_{mn}^{(1)} - \sum_{n=1}^N M_n J_{mn} - \sum_{n=1}^{N} M_n K_{mn}^{(1)}} \qquad EFIE-O 
\end{equation}
\begin{equation}\label{eqn:MoM MFIE-I}
\boxed{0 =  - \sum_{n=1}^N I_n J_{mn} + \sum_{n=1}^{N} I_n K_{mn}^{(2)} + \frac{1}{j \omega \mu_2} \sum_{n=1}^{N} M_n D_{mn}^{(2)} } \qquad MFIE-I
\end{equation}
\begin{equation}\label{eqn:MoM EFIE-I}
\boxed{0 =   \frac{1}{j \omega \varepsilon_2} \sum_{n=1}^{N} I_n D_{mn}^{(2)} + \sum_{n=1}^N M_n J_{mn} - \sum_{n=1}^{N} M_n K_{mn}^{(2)}} \qquad EFIE-I. 
\end{equation}
%
\par
Equations (\ref{eqn:MoM MFIE-O}), (\ref{eqn:MoM EFIE-O}), (\ref{eqn:MoM MFIE-I}) and (\ref{eqn:MoM EFIE-I}) look very similar in form to their counterparts (\ref{eqn:MFIE-O}), (\ref{eqn:EFIE-O}), (\ref{eqn:MFIE-I}) and (\ref{eqn:EFIE-I}). We can recast these equations in matrix form:
\begin{equation}\label{eqn:Matrix MFIE-O}
\boxed{\underline{V}^H =  \left[\underline{\underline{J}} + \underline{\underline{K}}^{(1)} \right] \underline{I} + \frac{1}{j \omega \mu_1}  \underline{\underline{D}}^{(1)} \underline{M} } \qquad MFIE-O 
\end{equation}
\begin{equation}\label{eqn:Matrix EFIE-O}
\boxed{\underline{V}^E =  \frac{1}{j \omega \varepsilon_1}  \underline{\underline{D}}^{(1)} \underline{I} -\left[\underline{\underline{J}} + \underline{\underline{K}}^{(1)} \right] \underline{M} } \qquad EFIE-O 
\end{equation}
\begin{equation}\label{eqn:Matrix MFIE-I}
\boxed{\underline{0} =  \left[-\underline{\underline{J}} + \underline{\underline{K}}^{(2)} \right] \underline{I} + \frac{1}{j \omega \mu_2}  \underline{\underline{D}}^{(2)} \underline{M} } \qquad MFIE-I
\end{equation}
\begin{equation}\label{eqn:Matrix EFIE-I}
\boxed{\underline{0} =  \frac{1}{j \omega \varepsilon_2}  \underline{\underline{D}}^{(2)} \underline{I}  + \left[\underline{\underline{J}} - \underline{\underline{K}}^{(2)} \right] \underline{M} } \qquad EFIE-I 
\end{equation}
where
\begin{itemize}
\item $D_{mn}^{(i)} = \int_{D_m}\vect{g}_m \arg{\vect{r}} \cdot \left( \left(\nabla \nabla \cdot + k_i^2\right) \int_{D_n} G_i\left(\vect{r}, \vect{r}'\right) \current{f}_n\left(\vect{r}'\right) d\vect{r}' \right) d\vect{r}$
\item $K_{mn}^{(i)} = \int_{D_m} \vect{g}_m \arg{\vect{r}} \cdot \left(\int_{D_n} \nabla G_i\left(\vect{r}, \vect{r}'\right) \times \current{f}_n\left(\vect{r}'\right) d\vect{r}'\right) d\vect{r}$
\item $J_{mn} = \int_{D_m} \frac{1}{2} \vect{g}_m \arg{\vect{r}} \cdot \vect{\hat{n}} \times \current{f}_{n}\arg{\vect{r}}d\vect{r}$
\item $V_m^{E} = -\int_{D_m}\vect{g}_m \arg{\vect{r}} \cdot \field{E}^\text{inc} d\vect{r}$
\item $V_m^{H} = -\int_{D_m}\vect{g}_m \arg{\vect{r}} \cdot \field{H}^\text{inc} d\vect{r}$.
\end{itemize}
%
\par
As for the PEC target, we also have that, for $K_{mn}$ and $J_{mn}$:
\begin{itemize}
\item $K_{mn} = 0$ if $\vect{g}_m$ and $\vect{f}_n$ are coplanar
\item $J_{mn} = 0$ if $\vect{g}_m = \vect{f}_m$, and $\vect{f}_n = \vect{f}_m$.
\end{itemize}

\section{Combining the MoM matrix equations}
%
\par
\subsection{PMCHWT}
%
\par
According to section \ref{subsec:PMCHWT}, for this formulation we combine the outside (1) and inside (2) on a per-field basis: EFIE-O (\ref{eqn:EFIE-O}) with EFIE-I (\ref{eqn:EFIE-I}), and MFIE-O (\ref{eqn:MFIE-O}) with MFIE-I (\ref{eqn:MFIE-I}). The testing is done with $\vect{g}_m = \vect{f}_m$. We have
\begin{equation}\label{eqn:Matrix EFIE}
\boxed{\underline{V}^{tE} =  \left[\frac{1}{j \omega \varepsilon_1}  \underline{\underline{D}}^{(1),t} + \frac{1}{j \omega \varepsilon_2}  \underline{\underline{D}}^{(2),t}\right] \underline{I} -\left[\underline{\underline{K}}^{(1),t} + \underline{\underline{K}}^{(2),t} \right] \underline{M} } \qquad EFIE 
\end{equation}
\begin{equation}\label{eqn:Matrix MFIE}
\boxed{\underline{V}^{tH} =  \left[\underline{\underline{K}}^{(1),t} + \underline{\underline{K}}^{(2),t} \right] \underline{I} + \left[\frac{1}{j \omega \mu_1}  \underline{\underline{D}}^{(1),t} + \frac{1}{j \omega \mu_2}  \underline{\underline{D}}^{(2),t}\right] \underline{M} } \qquad MFIE.
\end{equation}
%
\par
In matrix for, it is written as:
\begin{equation}
\left[
\begin{matrix}
  \frac{1}{j \omega \varepsilon_1} \underline{\underline{D}}^{(1), t} + \frac{1}{j \omega \varepsilon_2} \underline{\underline{D}}^{(2), t} & -\underline{\underline{K}}^{(1),t} - \underline{\underline{K}}^{(2),t}  \\ \\
  \underline{\underline{K}}^{(1),t} + \underline{\underline{K}}^{(2),t} & \frac{1}{j \omega \mu_1}\underline{\underline{D}}^{(1), t} + \frac{1}{j \omega \mu_2} \underline{\underline{D}}^{(2), t}
\end{matrix}
\right]
\cdot 
\left[
\begin{matrix}
  \underline{I} \\ \\
  \underline{M}  
\end{matrix}
\right]
=
\left[
\begin{matrix}
  \underline{V}^{tE} \\ \\
  \underline{V}^{tH}
\end{matrix}
\right].
\end{equation}



